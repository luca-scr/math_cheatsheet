% -*- mode: LaTeX; mode: flyspell; -*-
\documentclass[a4paper]{article}
\usepackage{amsthm}
\usepackage[T1]{fontenc}
\usepackage[estonian,english]{babel}
\usepackage[colorlinks, linkcolor=blue]{hyperref}
\usepackage[utf8]{inputenc}
\usepackage{xspace}
\usepackage{amsmath,amssymb}
\usepackage{pxfonts}
\usepackage{natbib}
\input{isomath}
\allowdisplaybreaks

\newcommand{\iid}{\textit{i.i.d.}\xspace}
\numberwithin{equation}{subsection}

\title{Mathematical Formulas and Other Valuable Knowledge\\
\normalsize that I have found Useful for Myself\\
and Decided to Write Down}
\author{Ott Toomet}
\date{Seattle, \today}


\begin{document}

\maketitle

\tableofcontents

\newpage
\section{Geometry}
\selectlanguage{estonian}

\subsection{Koordinaatteisendused}
\subsubsection{Pinnaelement koordinaatteisendusel}
Üleminekul koordinaatsüsteemilt $(x,y)$ uutele koordinaatidele
$(u=u(x,y), v=v(x,y))$ teisenevad väikesed koordinaatide sihilised
lõigud nii:
\begin{equation}
\begin{pmatrix}
\dif u\\
\dif v
\end{pmatrix} =
  \begin{pmatrix}
  u_x   & u_y \\
  v_x   & v_y
  \end{pmatrix}
  \begin{pmatrix}
  \dif x\\
  \dif y
  \end{pmatrix},
\end{equation}
kus paremal poolel esimene on vastavate osatuletiste maatriks
(jakobjaan $\mathcal{J}$).
Pinnaelement avaldub
\begin{equation}
\dif S = \dif x \dif y = 
 \frac{\dif u \dif v}{|\det \mathcal{J}|},
\end{equation}
kus
\begin{math}
|\det \mathcal{J}| = \dif u \dif v \sin( \widehat{u,v})
\end{math}
on kahe lühikese lõigu $\dif u$ ja $\dif v$ poolt defineeritud
elementaarpindala.  Analoogiline seos kehtib ka kõrgemate mõõtmete
puhul.


\subsection{Kolmnurk}
\subsubsection{Seos kolmnurga külgede ja nurkade vahel}

\begin{equation}
a^2 + b^2 - 2ab \cos \gamma = c^2.
\end{equation}
Täisnurksel kolmnurgal, kui $\gamma = 90\kraad$, kehtib
\emph{Pythagorase teoreem}:
\begin{equation}
a^2 + b^2 = c^2.
\end{equation}

\subsubsection{Kolmnurga pindala}
Kui kolmnurk on tasandil antud kolme punktiga $(0,0)$, $(X_A, Y_A)$
ning $(X_B, Y_B)$ siis kolmnurga pindala on
\begin{equation}
S = \frac{1}{2} \left| X_A \cdot Y_B - Y_A \cdot X_B \right| =
\frac{1}{2} \abs
  \begin{vmatrix}
  X_A   & Y_A \\
  X_B   & X_B
  \end{vmatrix}.
\end{equation}
Tõestus: joonista kolmnurk välja ja vaata, millised pindalad kirjeldab
determinant.


\subsection{Muud}
\label{sec:geomeetria_muud}

\subsubsection{Joone kõverus}

Parameetriliselt antud joone kõverus:
\begin{equation}
  \label{eq:param_joone_kqverus}
  k = \frac{x'y'' - y'x''}{({x'}^2 + {y'}^2)^{3/2}}
\end{equation}
\selectlanguage{english}


\newpage

\section{Functions}
\label{sec:functions}

\subsection{Algebraic Functions}

\paragraph{Gamma Function $\Gamma(\cdot)$}
\begin{equation}
  \Gamma(x) = \int_{0}^{\infty} t^{x-1} \me^{-1}\,\dif t
\end{equation}
Properties:
\begin{align}
  \Gamma(n) &= (n-1)!
  \\
  \Gamma(n+1) &= n \Gamma(n)
\end{align}

\paragraph{Modified Bessel Function of the First Kind $I_{\alpha}(\cdot)$}
\label{sec:modified_bessel_function}
\begin{equation}
  I_{\alpha}(x) =
  \sum_{k=0}^{\infty} 
  \frac{1}{k! (\alpha + k)!}
  \left( \frac{x}{2} \right)^{2k + \alpha}
\end{equation}


\newpage
\section{Calculus}
\label{sec:calculus}

\subsection{Logarithm}

Properties
\begin{equation}
  \log x^\alpha = \alpha \log x 
  \qquad
  \text{and}
  \qquad
  \log( xy) = \log x + \log y
\end{equation}


\subsection{Limits}
\label{sec:limits}

\begin{equation}
  \lim_{x\to 0} x \log x = 0
\end{equation}
\begin{proof}
  Write $x \log x = (\log x)/(1/x)$ and use L'Hospital's rule.
\end{proof}

\begin{equation}
  \label{eq:phi(x)/Phi(x)}
  \lim_{x\to \infty}
  \frac{1}{x} \frac{\phi(x)}{1 - \Phi(x)} = 1,
\end{equation}
where $\phi(\cdot)$ and $\Phi(\cdot)$ are normal density and
distribution functions.

\begin{proof}
  The fraction can be written as
  \begin{equation*}
    \frac{\phi(x)}{1 - \Phi(x)}
    =
    \frac{\phi(x)}{\int_{x}^{\infty} \phi(s) \dif s}.
  \end{equation*}
  The integral can be expressed as the Rieman limit
  \begin{equation*}
    \int_{x}^{\infty} \phi(s) \dif s
    =
    \lim_{\delta \to 0} 
    \big[
    \phi(x) \delta +
    \phi(x + \delta) \delta +
    \phi(x + 2 \delta) \delta +
    \dots 
    \big].
  \end{equation*}
  Using the expression for $\phi(\cdot)$ we get
  \begin{equation*}
    \phi(x + \delta) = \phi(x) \me^{-x \delta} \me^{-\delta^{2}/2}
  \end{equation*}
  and hence we may write the denominator in \eqref{eq:phi(x)/Phi(x)}
  as
  \begin{equation*}
    x \phi(x)
    \lim_{\delta \to 0} 
    \big[
    1 +
    \me^{-x \delta} \me^{-\delta^{2}/2} +
    \me^{-2 x \delta} \me^{-4 \delta^{2}/2} +
    \me^{-3 x \delta} \me^{-9 \delta^{2}/2} +
    \dots 
    \big] \delta.
  \end{equation*}
  This expressions $\me^{-x \delta}$, $\me^{-2 x \delta}$ and so on
  for a geometric sequence with sum $1/(1 - \me^{-x \delta}) \approx 1/x
  \delta$ as $x \delta \to 0$.  Accordingly, we let $\delta \to 0$ and $x
  \to \infty$ in such as way that $x \delta \to 0$.  We have to show
  that the other terms $\me^{n^{2} \delta^{2}/2}$ do not ``disturb''
  the geometric sequence too much.

  Now find $n^{*}$, starting of which the residual sum on the
  geometric sequence $1 + \me^{-x \delta} + \me^{-2 x \delta} + \dots$
  is less than $\varepsilon > 0$:
  \begin{equation*}
    \sum_{n=n^{*}}^{\infty} \me^{-n x \delta} < \varepsilon
    \qquad\Rightarrow\qquad
    n^{*} > -\frac{\log \varepsilon + \log(1 - \me^{-x \delta})}{x \delta}
  \end{equation*}
  choose $n^{**} > -\frac{\log \varepsilon}{x \delta} + 1 > n^{*}$.
  Now find 
  \begin{equation*}
    \exp\frac{{n^{**}}^{2} \delta^{2}}{2}
    =
    \exp\left(
      \frac{\log^{2}\varepsilon}{2 x^{2}}
      +
      \delta\frac{\log\varepsilon}{x}
      +
      \frac{\delta^{2}}{2}
    \right).
  \end{equation*}
  Because all the terms in parenthesis converge to as $x\to\infty$ and
  $\delta \to 0$, the exponent converges to 1.  Hence, at the limit, the
  Rieman sum is solely determined by the geometric sequence, and we
  have denominator in \eqref{eq:phi(x)/Phi(x)} equal to $\phi(x)$.
\end{proof}

\subsection{Differentiation}

\subsubsection{Simple Derivatives}

\begin{align}
  \pderiv{x} a^{x} &= a^{x} \log a
  \\
  \tan \phi' &= \frac{1}{\cos^2 \phi} 
  \\
  \arctan x' &= \frac{1}{1+x^2}
\end{align}


\subsubsection{Directional Derivative}

\begin{equation}
  f'(\vec{x}; \vec{u})
  =
  \lim_{h \to 0}
  \frac{f(\vec{x} + h\vec{u}) - f(\vec{x})}
  {h}
\end{equation}


\subsubsection{Jacobian Matrix}
\label{sec:jacobian_matrix}

\begin{equation}
  \mathcal{J} = 
  \begin{bmatrix}
    \dfrac{\partial \mathbf{f}}{\partial x_1} & \cdots &
    \dfrac{\partial \mathbf{f}}{\partial x_n} 
  \end{bmatrix}
  = 
  \frac{\partial (f_1, \dots, f_n)}
  {\partial (x_1, \dots, x_n)}
  =
  \vec{D}\,\vec{f}(\vec{x})
  =
  \begin{bmatrix}
    \dfrac{\partial f_1}{\partial x_1} & \cdots & \dfrac{\partial f_1}{\partial x_n}\\
    \vdots & \ddots & \vdots\\
    \dfrac{\partial f_m}{\partial x_1} & \cdots & \dfrac{\partial
      f_m}{\partial x_n} 
  \end{bmatrix}
\end{equation}


\selectlanguage{estonian}

\subsubsection{Ümbrikuteoreem  (\emph{{envelope theorem}})}
\label{sec:envelope_theorem}

Olgu $M$ defineeritud kui optimum funktsioonist $f$:
\begin{equation}
  M(a) = \max_{x} f(x, a),
\end{equation}
kus $a$ on parameeter.  Siis
\begin{equation}
  \frac{\dif M(a)}{\dif a} =  \frac{\partial f(x^*, a)}{ \partial a} \Bigg|_{x^* = x(a)}.
\end{equation}



\selectlanguage{english}


\subsubsection{Normal Density Related Derivatives}

\paragraph{One-Dimensional Case}
\begin{eqnarray}
\frac{\dif}{\dif x} \Phi(-x) &=&
  \phi(x) \\
%
\frac{\dif}{\dif x} 
\frac{1}{\sigma} 
\phi \left( \frac{x-\mu}{\sigma} \right) &=&
  - \frac{1}{\sigma} 
    \left( \frac{x-\mu}{\sigma} \right)
    \phi \left( \frac{x-\mu}{\sigma} \right)
\\
\frac{\dif}{\dif \sigma} 
\frac{1}{\sigma} 
\phi \left( \frac{x-\mu}{\sigma} \right) &=&
    \frac{1}{\sigma^{2}} 
    \left[ \left( \frac{x-\mu}{\sigma} \right)^{2} - 1 \right]
    \phi \left( \frac{x-\mu}{\sigma} \right)
\end{eqnarray}

\paragraph{Multi-Dimensional Case}

Let 
\begin{math}
  \vec{x} = \left(
    \begin{array}{r}
      x_{1} \\ x_{2}
    \end{array}
  \right),
\end{math}
$\mat{\Sigma} = \left[
 \begin{array}{rr} 1 & \rho \\
 \rho & 1
\end{array}\right]$ be the 2-dimensional variance-covariance matrix
and $\phi(\cdot,\cdot)$ the 2-D normal density, defined
in~\eqref{eq:N-Dimensional_normal}:
\begin{align}
  \pderiv{x_{1}}
  \phi(\vec{x}, \mat{\Sigma})
  &=
  \phi(\vec{x}, \mat{\Sigma})
  \frac{\rho x_{2} - x_{1}}{1 - \rho^{2}}
  \\
  \pderiv{\rho}
  \phi(\vec{x}, \mat{\Sigma})
  &=
  \phi(\vec{x}, \mat{\Sigma})
  \Bigg[
  \frac{\rho}{1 - \rho^{2}}
  \left(
    1 - \vec{x}' \mat{\Sigma}^{-1} \vec{x}
  \right)
  +
  \frac{x_{1} x_{2}}{1 - \rho^{2}}
  \Bigg]
  =
  \notag\\
  &=
  \phi(\vec{x}, \mat{\Sigma})
  \Bigg[
  \frac{\rho}{1 - \rho^{2}}
  -
  \frac{\rho}{(1 - \rho^{2})^{2}}
  (x_{1}^{2} - 2 \rho x_{1} x_{2} + x_{2}^{2})
  +
  \frac{x_{1} x_{2}}{1 - \rho^{2}}
  \Bigg]
  \\
  \frac{\partial^{2}}{\partial x_{1} \partial x_{2}}
  \phi(\vec{x}, \mat{\Sigma})
  &=
  \phi(\vec{x}, \mat{\Sigma})
  \left[
    \frac{(x_{1} - \rho x_{2})(x_{2} - \rho x_{1})}
    {(1 - \rho^{2})^{2}}
    +
    \frac{\rho}{1 - \rho^{2}}
  \right]
\end{align}


\subsubsection{Derivatives of Gamma Function}
\begin{eqnarray}
\Gamma'(x) &=&
  \psi(x) \Gamma(x)\\
\Gamma''(x) &=&
  \Gamma(x) \left[ \psi'(x) + \psi^2(x) \right],
\end{eqnarray}
where $\psi(x)$ is the digamma function.


\subsubsection{Differentiation of Sums}
\begin{align}
  \pderiv{x_{i}}\sum_{j} x_{j} &= 1
  \\
  \pderiv{x_{i}} 
  \left( \sum_{j} x_{j} \right)^{2}
  &= 2 \sum_{j} x_{j}
\end{align}


\subsubsection{General Differentiation Rules}

\selectlanguage{estonian}
\paragraph{Pöördfunktsiooni tuletis}
\begin{equation}
\frac{\dif}{\dif x} f^{-1}(x) = 
  \left. \frac{1}{\frac{\dif}{\dif x} f(x)} \right|_{x=f^{-1}(y)} =
  \frac{1}{f'[f^{-1}(y)]}
\end{equation}

\paragraph{Mitmekihilise (liit-) funktsiooni tuletis}
Olgu $v=v(b)$ ja $f(v) = f(v(b)) = g(b)$.  Tuletised:
\begin{eqnarray}
  \label{eq:mitmekihilise_tuletis}
  \frac{\partial g}{\partial b} &=& 
  \frac{\partial f}{\partial v} \cdot \frac{\partial v}{\partial b}
  \equiv
  f'(v(b)) \cdot v'(b)\\
                                %
  \frac{\partial^2 g}{\partial b^2} &=&
  \frac{\partial^2 g}{\partial v^2} 
  \left(\frac{\partial v}{\partial b} \right)^2 +
  \frac{\partial f}{\partial v}\frac{\partial^2 v}{\partial b^2}
\end{eqnarray}



\selectlanguage{english}
\subsection{Integration}

\subsubsection{Simple Integrals}
\label{sec:simple_integrals}

\begin{equation}
  \int x^{\alpha} = \frac{1}{\alpha + 1} x^{\alpha+1},
  \quad \alpha \not= -1
\end{equation}

\paragraph{Leibnitz' Rule}
\begin{eqnarray}
  \pderiv{x}
  \int_{a(x)}^{b(x)} f(y) \;\dif y
  & = &
  f[b(x)] b'(x) - f[a(x)]a'(x)\\
%
  \pderiv{x}
  \int_{a(x)}^{b(x)} f(y,x) \;\dif y 
  & = &
  f[b(x),x] b'(x) - f[a(x),x]a'(x) + \notag\\
  & + & 
  \int_{a(x)}^{b(x)} \frac{\partial}{\partial x} f(y,x) \;\dif y
\end{eqnarray}



\subsubsection{Integrals related to probability distributions}

\paragraph{Normal distribution}
Let
\begin{equation}
  \phi(x)
  =
  \frac{1}{\sqrt{2\pi}}
  \me^{-\frac{1}{2} x^{2}}.
\end{equation}
To prove the following, simply integrate $\phi(\cdot)$, in most cases
by parts.
\begin{align}
  \int \phi(x) \,\dif x
  &\equiv
  \Phi(x)
  \\
  \int_{a}^{b} \phi\left(\frac{x - \mu}{\sigma} \right) \,\dif x
  &=
  \sigma \Phi \left(\frac{b - \mu}{\sigma} \right) -
  \sigma \Phi \left(\frac{a - \mu}{\sigma} \right)
  \\
  \int \phi^2(x) \,\dif x
  &=
  \frac{1}{2\mpi} \Phi(\sqrt{2} x)
  \\
  \int x \phi(x) \,\dif x
  &=
  -\phi(x)
  \\
  \int_{-\infty}^\infty x \phi \left( \frac{x-\mu}{\sigma} \right) 
  \dif x 
  &=
  \sigma \mu \\
  % integral x phi(.)
  \int  x \phi \left( \frac{x-\mu}{\sigma} \right) \dif x 
  &=
  \sigma \mu
  \Phi \left( \frac{x-\mu}{\sigma} \right) 
  -
  \sigma^2
  \phi \left( \frac{x-\mu}{\sigma} \right) 
  + C
  \\
  %
  \int_a^b  x \phi \left( \frac{x-\mu}{\sigma} \right) \dif x 
  &=
  \sigma \mu \left[
    \Phi \left( \frac{b-\mu}{\sigma} \right) -
    \Phi \left( \frac{a-\mu}{\sigma} \right) \right] + \notag\\
  &+
  \sigma^2 \left[
    \phi \left( \frac{a-\mu}{\sigma} \right) -
    \phi \left( \frac{b-\mu}{\sigma} \right) \right]\\
  % integral x^2 phi(.)
  \int x^2 \phi (x) \, \dif x 
  &= 
  -x \phi (x) + \Phi(x)\\
  % 
  \int_a^b x^2 \phi(x) \,\dif x
  &=
  a\phi(a) - b\phi(b) + \Phi(b) - \Phi(a)\\
  %
  \int x^2 \phi \left( \frac{x}{\sigma} \right) \, \dif x 
  &= 
  -\sigma^2 x \phi \left( \frac{x}{\sigma} \right) +
  \sigma^3 \Phi \left( \frac{x}{\sigma} \right)\\
  % 
  \int_{-\infty}^\infty x^2 \phi \left( \frac{x-\mu}{\sigma} \right)
  \dif x 
  &= 
  \sigma ( \mu^2 + \sigma^2) \\
  % 
  \int_{-\infty}^\infty \me^{\alpha x} 
  \phi \left( \frac{x-\mu}{\sigma} \right) \,\dif x 
  &= 
  \me^{\alpha \mu + \frac{1}{2} \sigma^2 \alpha^2} \\
  % 
  \int_s^t \phi(u) \log\phi(u) \,\dif u 
  &=
  \frac{1}{2}\left[ \Phi(s) - \Phi(t)\right] \left(1+\log 2\pi \right)
  + \frac{1}{2}\left[ t\phi(t) - s\phi(s)\right] 
  \nonumber\\
  &\\
  % 
  \int_s^t \phi(u) \log\phi(u - \mu) \,\dif u 
  &=
  \frac{1}{2}\left[ \Phi(s) - \Phi(t)\right]
  \left( 1 + \log 2\pi + \mu^2 \right) + 
  \nonumber\\
  & + \frac{1}{2}\left[ t\phi(t) - s\phi(s)\right] +
  \mu \left[ \phi(s) - \phi(t) \right]\\
  % 
  \int x \phi^2(x) \,\dif x
  &=
  -\frac{1}{2} \phi^2(x)\\
  %
  \int \phi(x) \Phi(x) \,\dif x
  &=
  \frac{1}{2} \Phi^2(x)\\
  %
  \int x \phi(x) \Phi(x) \,\dif x
  &=
  -\phi(x) \Phi(x) + \frac{1}{2\sqrt{\mpi}} \Phi(\sqrt{2}\,x)\\
  \int x^2 \phi(x) \Phi(x) \,\dif x
  &=
  \frac{1}{2} \Phi^2(x)
  -x \phi(x) \Phi(x)
  -\frac{1}{2} \phi^2(x)
\end{align}


\paragraph{Log-normal density}

Let $f(\cdot)$ be the log-normal density.
\begin{multline}
\int_a^\infty x f(x) \,\dif x
  =
  \int_a^b
  \frac{1}{\sqrt{2\pi}\sigma}
  \exp\left[
    -\frac{1}{2}
    \left(
      \frac{\log x - \mu}{\sigma}
    \right)^2
  \right]
  \,\dif x
  =
  \\
  =
  \frac{1 - \Phi \left(
      \frac{\log a - \mu - \sigma^2}{\sigma}
      \right)}
    {1 - \Phi \left(
      \frac{\log a - \mu}{\sigma}
      \right)}
    \me^{\mu + \frac{1}{2}\sigma^2} 
\end{multline}
\begin{multline}
  \int_a^b x f(x) \,\dif x
  =
  \int_a^b
  \frac{1}{\sqrt{2\pi}\sigma}
  \exp\left[
    -\frac{1}{2}
    \left(
      \frac{\log x - \mu}{\sigma}
    \right)^2
  \right]
  \,\dif x
  =
  \\
  =
  \me^{\mu + \frac{1}{2}\sigma^2}
  \left[
    -\Phi \left(
      \frac{\log a - \mu - \sigma^2}{\sigma}
    \right)
    +\Phi \left(
      \frac{\log b - \mu - \sigma^2}{\sigma}
    \right)
  \right]
\end{multline}


\subsubsection{Other integrals}

\begin{align}
  \int_0^t \me^{-rT} \,\dif T
  &=
  \frac{1}{r}
  \left[
    1 - \me^{-rt}
  \right]
  \\
  \int \frac{\dif x}{(p + q^{ax})^2} \,\dif x 
  &=
  \frac{x}{p^2} +
  \frac{1}{ap(p + q\me^{ax})} -
  \frac{1}{ap^2} \ln (p + q\me^{ax})
  \\
  \int \log x \,\dif x 
  &=
  x \log x - x
  \\
  \int \me^{-ax} \dif x 
  &=
  \frac{1}{a} \left[ 1 - \me^{-ax} \right]\\
                                %
  \int x \me^x \dif x 
  &=
  x \me^x - \me^x\\
                                %
  \int x \me^{-x} \dif x
  &=
  -x \me^{-x} - \me^{-x}
  \\
  \int x \me^{\alpha x} \,\dif x
  &=
  \frac{x}{\alpha} \me^{\alpha x} - \frac{1}{\alpha^2} \me^{\alpha x}
  \\
  \int_a^b x \me^{\alpha x} \,\dif x
  &=
  \frac{1}{\alpha}
  \left[
    \me^{\alpha b} 
    \left(b - \frac{1}{\alpha}\right)
    -
    \me^{\alpha a} 
    \left(a - \frac{1}{\alpha}\right)
  \right]
  \\
  \int x^2 \me^{\alpha a} \,\dif x
  &=
  \frac{1}{\alpha} x^2 \me^{\alpha x} -
  \frac{2}{\alpha} \int x \me^{\alpha x} \,\dif x
  \\
  \int_a^b x^2 \me^{\alpha a} \,\dif x
  &=
  \frac{1}{\alpha}
  \left[
    \me^{\alpha b}
    \left(
      b^2 - \frac{2b}{\alpha} + \frac{2}{\alpha^2}
    \right)
    -
    \me^{\alpha a}
    \left(
      a^2 - \frac{2a}{\alpha} + \frac{2}{\alpha^2}
    \right)
  \right]
  \\
  \int_0^\infty x^\alpha \me^{-\beta x} \dif x 
  &=
  \frac{\Gamma(\alpha + 1)}{\beta^{\alpha+1}}
\end{align}
Let $f(\cdot)$ be a distribution function and $\bar F(\cdot)$ the
corresponding survival function:
\begin{equation}
  \int_c^b \left[
    f(x)
    \int_c^x w(y) \,\dif y
  \right]
  \,\dif x
  =
  \int_c^b
  \bar F(x) w(x) \,\dif x
\end{equation}


\selectlanguage{estonian}
\paragraph{Euleri konstant}
\begin{equation}
\int_0^\infty e^{-z} \log z \,\dif z = c \approx -0,5772 \qquad
\end{equation}


\subsubsection{Üldised integreerimise reeglid}
\paragraph{Muutuja vahetus integraali all}
Olgu vaja üle minna muutujatelt $(x_1 \ldots, x_N)$ muutujatele $(y_1,
\ldots, y_N)$.  Sel juhul
\begin{multline}
\int_V f(x_1, \ldots, x_N) \dif x_1 \ldots \dif x_N =\\
%
  = \int_V f[ y_1(x_1, \ldots, x_N), \ldots, y_N(x_1, \ldots, x_N) ]
    \frac{\dif y_1 \ldots \dif y_N}{|\mathcal{J}|} =\\
%
  = \int_V g(y_1, \ldots, y_N) 
    \frac{\dif y_1 \ldots \dif y_N}{|\mathcal{J}|},
\end{multline}
kus
\begin{equation}
|\mathcal{J}| = \left| \frac{\partial(y_1, \ldots, y_N)}
  {\partial(x_1, \ldots, x_N)} \right|
\end{equation}
on koordinaatteisenduse jakobjaani absoluutväärtus.


\subsubsection{Analüütilised funktsioonid}

\paragraph{Gammafunktsioon}
\begin{equation}
\Gamma(p) = 
  \int_0^\infty \lambda^{p-1} e^{-\lambda} \dif\lambda
\end{equation}
Gammafunktsiooni omadus:
\begin{equation}
\Gamma(p+1) = p \Gamma(p) = p!
\end{equation}
Tõestus: integreeri ositi.

\paragraph{Digamma funktsioon}
\begin{equation}
\psi(\alpha) = \frac{d \log \Gamma(\alpha)}{d \alpha} =
  \frac{1}{\Gamma(\alpha)}
  \int_0^\infty x^{\alpha-1} e^{-x} \log x dx.
\end{equation}
Digamma omadus:
\begin{equation}
\psi(\alpha+1) = \frac{1}{\alpha} + \psi(\alpha).
\end{equation}
Digamma arvväärtused:
\begin{equation}
\psi(1) = -0,5772 \qquad \psi(2) = 0,4228
\end{equation}

\paragraph{Beetafunktsioon}
\begin{equation}
B(p,q) = \frac{\Gamma(p) \Gamma(q)}{\Gamma(p+q)} =
  \int_0^1 x^{p-1} (1-x)^{q-1} dx,
\end{equation}
kusjuures $p>0$ ja $q>0$.


\subsubsection{Laplace'i teisendus}
Laplace'i teisendus juhusliku muutuja $X$ jaotusfunktsioonist on
\begin{equation}
L_f (s) = \E \me^{-sX} = 
  \int \me^{-sx} dF_X (x).
\end{equation}
Laplace'i teisendus on sama mis momendifunktsioon.


\subsubsection{Numbriline integreerimine}

\paragraph{Monte-Carlo integraal}

Olgu vaja leida
\begin{equation}
  I = \int_a^b f(x)\, \dif x =
  (b - a) \int_a^b f(x) \frac{1}{b - a}\, \dif x =
  (b - a) \E [f(X)],
\end{equation}
kus $X \sim U(a, b)$.  Valimis suurusega $N$ olgu $x_1, \ldots x_N
\sim i.i.d U(a, b)$.  Siis integraali hinnanguks on funktsiooni
väärtuste keskmine ja veahinnanguks tema standardhälve valimis:
\begin{eqnarray}
  \hat I_N &=& (b - a) \frac{1}{N} \sum_{i=1}^N f(x_i)\\
                                %
  \widehat{\var \hat I_N} &=&
  \frac{(b - a)^2}{N} \frac{1}{N} \sum_{i=1}^N 
  \left[f(x_i) - \frac{1}{N} \hat I_N \right]^2
\end{eqnarray}



\clearpage

\subsection{Differential Equations}
\label{sec:differential_equations}

Let $c$ be a constant.
\begin{align}
  \begin{split}
    y(x)' + P y &= Q
    \\
    y(x) &= \frac{Q}{P} + c \me^{-Px}
  \end{split}
  \\
  \begin{split}
    y(x)' + P(x) y &= Q(x)
    \\
    y(x) &= \me^{-\int P(x)\,\dif x}
    \int Q(x) \me^{\int P(x)\,\dif x} \,\dif x
    + c \me^{-\int P(x)\,\dif x}
  \end{split}
\end{align}



\clearpage
\subsection{Optimization}

\subsubsection[Second-order conditions]{Second-order maximum/minimum conditions for constrained
  optimisation}

The problem is
\begin{equation}
  \begin{split}
    \max z = f(x_1, x_2, \dots, x_n)\\
    \text{s.t. } 
    g(x_1, x_2, \dots, x_n) = 0.
  \end{split}
\end{equation}
Corresponding Lagrangian is
\begin{equation}
  Z = f(x_1, x_2, \dots, x_n) - \lambda g(x_1, x_2, \dots, x_n).
\end{equation}
Corresponding \emph{bordered Hessian} is:
\begin{equation}
  \begin{vmatrix}
    \bar{\mat H}
  \end{vmatrix}
  =
  \begin{vmatrix}
    0   & g_1    & g_2    & \dots & g_n\\
    g_1 & Z_{11} & Z_{12} & \dots & Z_{1n}\\
    g_2 & Z_{21} & Z_{22} & \dots & Z_{2n}\\
    \vdots & \vdots & \vdots & \ddots & \vdots \\
    g_n & Z_{n1} & Z_{n2} & \dots & Z_{nn}
  \end{vmatrix}
\end{equation}
and successive \emph{principal minors} are:
\begin{equation}
  \begin{vmatrix}
    \bar{\mat{H}}_2
  \end{vmatrix}
  =
  \begin{vmatrix}
    0   & g_1    & g_2\\
    g_1 & Z_{11} & Z_{12}\\
    g_2 & Z_{21} & Z_{22}
  \end{vmatrix}
  \qquad
  \begin{vmatrix}
    \bar{\mat{H}}_3
  \end{vmatrix}
  =
  \begin{vmatrix}
    0   & g_1    & g_2    & g_3\\
    g_1 & Z_{11} & Z_{12} & Z_{13}\\
    g_2 & Z_{21} & Z_{22} & Z_{23}\\
    g_3 & Z_{31} & Z_{32} & Z_{33}
  \end{vmatrix}
  \qquad
  \dots
  \qquad
  \begin{vmatrix}
    \bar{\mat{H}}_n 
  \end{vmatrix}
  =
  \begin{vmatrix}
    \bar{\mat H}
  \end{vmatrix}
\end{equation}
The second derivative $\dif^2 z$ is positive definite iff
\begin{equation*}
  \bar{\mat{H}}_2 < 0,
  \quad
  \bar{\mat{H}}_3 < 0,
  \quad
  \dots,
  \quad
  \bar{\mat{H}}_n < 0,
  \quad
\end{equation*}
and negavitve definite iff
\begin{equation*}
  \bar{\mat{H}}_2 > 0,
  \quad
  \bar{\mat{H}}_3 < 0,
  \quad
  \bar{\mat{H}}_n > 0,
  \quad
  \dots
  \quad
\end{equation*}
Note that $\bar{\mat{H}}_1$ is always negative.

Proof: \citet{chiang1984}.



\newpage
\subsubsection{Optimal control (dynamic optimisation)}

The problem is:
\begin{equation}
  \begin{split}
    \max_u & \int_0^T F(t, y, u) \,\dif t\\
    \text{s.t. }
    \dot y & = f(t, y, u)\\
    y(0) & = A
    \qquad
    y(T) \text{ free}.
  \end{split}
\end{equation}
Corresponding \emph{Hamiltonian} is
\begin{equation}
  H(t, y, u, \lambda)
  \equiv
  F(t, y, u) +
  \lambda(t) f(t, y, u).
\end{equation}
The first order conditions for optimum are
\begin{enumerate}
\item $\max_u H(t, y, u, \lambda) \quad\forall t \in [0,T]$ or, less
  generally, $\pderiv[H]{t} = 0$ (optimality condition).
\item $\dot y = \pderiv[H]{\lambda}$ (equation of motion for $y$).
\item $\dot \lambda = - \pderiv[H]{y}$ (equation of motion for $\lambda$).
\item $\lambda(T) = 0$ (transversality condition).
\end{enumerate}

Proof: \citet{miller1979}



\subsubsection{Newton-Raphson maximization}
\label{sec:Newton-Raphson}

Non-linear continuous function of $N$-dimensional parameter can,
under suitable assumptions, be approximated as $N$-dimensional parabola.
When running non-linear maximization, we may approximate the function
in this way at the initial value of the parameter vector.  The maximum
of the approximation can be used as the initial value for the next
step.

Let us maximise a function $l(\vec{\vartheta})$ where $\vec{\vartheta}$
is a $N$-dimesional parameter vector.  Let the initial value of the
parameter be $\vec{\vartheta}_0$.  From Taylor's approximation:
\begin{equation}
  l(\vec{\vartheta}) \approx
  l(\vec{\vartheta}_0) 
  + 
  \left.
    \frac{\partial l(\vec{\vartheta})}
    {\partial\vec{\vartheta}}
  \right|_{\vec{\vartheta} = \vec{\vartheta}_0}
  (\vec{\vartheta} - \vec{\vartheta}_0) 
  +
  \frac{1}{2}
  (\vec{\vartheta} - \vec{\vartheta}_0)'
  \left.
    \frac{\partial^2 l(\vec{\vartheta})}
    {\partial\vec{\vartheta} \partial\vec{\vartheta}'}
  \right|_{\vec{\vartheta} = \vec{\vartheta}_0}
  (\vec{\vartheta} - \vec{\vartheta}_0)
\end{equation}
At the maximum $\partial l(\vec{\vartheta}) /
\partial\vec{\vartheta} = 0$ and hence the parameter value at the
maximum (the initial value for the next iteration):
\begin{equation}
  \vec{\vartheta}_1 = 
  \vec{\vartheta}_0 -
  \left[
    \left.
      \frac{\partial^2 l(\vec{\vartheta})}
      {\partial\vec{\vartheta} \partial\vec{\vartheta}'}
    \right|_{\vec{\vartheta} = \vec{\vartheta}_0}
  \right]^{-1}
  \left.
    \frac{\partial l(\vec{\vartheta})}
    {\partial\vec{\vartheta}}
  \right|_{\vec{\vartheta} = \vec{\vartheta}_0}
\end{equation}

The algorith requires either programming the analytical Hessian matrix
$\frac{\partial^2 l(\vec{\vartheta})} {\partial\vec{\vartheta}
  \partial\vec{\vartheta}'}$, or calculating the Hessian matrix by
numeric differentiation.  The first way may be complicated, the latter
one slow and subject to numerical errors.


\paragraph{BHHH maximization}

BHHH is a particular version of Newton-Raphson algorithm, suitable for
maximizing log-likelihood function only.  BHHH uses the information
equlity for approximating the Hessian:
\begin{equation}
  \E
  \left[
    \frac{\partial^2 l(\vec{\vartheta})}
    {\partial\vec{\vartheta} \partial\vec{\vartheta}'}
  \right]_{\vec{\vartheta} = \vec{\vartheta}_0}
  =
  - \E
  \left[
    \left.
      \frac{\partial l(\vec{\vartheta})}
      {\partial\vec{\vartheta}'}
    \right|_{\vec{\vartheta} = \vec{\vartheta}_0}
    % 
    \left.
      \frac{\partial l(\vec{\vartheta})}
      {\partial\vec{\vartheta}}
    \right|_{\vec{\vartheta} = \vec{\vartheta}_0}
  \right]
\end{equation}
This algorithm does not require Hessian matrix (this is approximated).
However, it typically requires around 10 times more iterations for
convergence as the approximation may be quite imprecise when initial
values are far off the target.  Note also that while the estimates are
exactly the same as in the case of NR algorithm, the standard errors
may be different on a finite sample \citep{calzolari+fiorentini1993}.






\newpage
\selectlanguage{estonian}
\section{Algebra}

\subsection{Mõisted}
\label{sec:algebra_mqisted}

\begin{description}
\item[proper subset] $A$ on $B$ \emph{proper subset} kui $A\subseteq
  B$ kui $B \not\subseteq A$.
\item[proper subspace] Kui $A$ ja $B$ ja $A$ on $B$ \emph{proper
    subset}.  Näiteks tasandi tõeline (lineaarne) alamruum on sirge.
\end{description}

\subsection{Tehted hulkadega}
\begin{eqnarray}
\left( \bigcap_{A \in F} A \right)^C & = &
  \bigcup_{A \in F} A^C\\
%
\left( \bigcup_{A \in F} A \right)^C & = &
  \bigcap_{A \in F} A^C\\
%
A \setminus B & = & A \cap B^C
\end{eqnarray}

\selectlanguage{english}
\subsection{Simple Algebra}

\paragraph{Binomial Theorem}
\begin{equation}
(a+b)^n = \sum^n_{i=0} \binom{n}{i} \, a^{n-i} b^i
\end{equation}
where $\binom{n}{i}$ is the \emph{binomial coefficient}, number of
distinct combinations of $i$ elements out of $n$ elements in total:
$\binom{n}{i} \equiv C_n^i = \frac{n!}{(n-i)! i!}$.
\begin{align}
  \sum_{i=0}^{n} \binom{n}{i}\: \me^{\alpha i}
  &=
  (1 + \me^{\alpha})^{n}
  \\
  \sum_{i=0}^{n} i \, \binom{n}{i}\: \me^{\alpha i}
  &=
  n \me^{\alpha}(1 + \me^{\alpha})^{n-1}
\end{align}


\selectlanguage{estonian}
\paragraph{Geomeetrilise jada summa}
\begin{equation}
S=1+q+q^2+q^3+... = \frac{1 }{ 1-q}.
\end{equation}
Tõestus: kirjuta välja $qS$, lahuta ja avalda $S$. Märkus: kui jada on
kujul $S'=q+q^2+q^3+...$, siis $S'=qS$. Oluline erijuht kui $q=\frac{1 }{
1+r}$:
\begin{equation}
 S = 1 + \frac{1 }{ 1+r} + \frac{1 }{ (1+r)^2} + ... = 1 + \frac{1 }{ r}.
\end{equation}
\begin{equation}
  \sum_{i=0}^\infty i p^i = \frac{p}{(1 - p)^2}
\end{equation}
Tõestus: kui $S$ on antud summa, siis avalda $S - pS$ \ldots

\paragraph{Eksponent piirväärtusena}
\begin{equation}
 \lim_{n\to\infty} \left( 1 + \frac{a}{ n} \right)^n = e^a.
\end{equation}
Tõestus: arenda Newtoni binoomvalemiga ritta, arvesta
\eqref{eq:faktoriaalide jagatis}.

\paragraph{faktoriaalide jagatis}
\begin{equation}
\lim_{n\to\infty} \frac{n! }{ (n-m)!} = n^m
\label{eq:faktoriaalide jagatis}
\end{equation}
Märkus: siin on eeldatud, et $m \not\to \infty$. Jaga läbi, arvesta, et
$n-1 \approx n$. Seose erijuht:
\begin{equation}
 \lim_{n\to\infty} \frac{n}{m} = \lim_{n\to\infty} \frac{n! }{ (n-m)! m!}
= \frac{n^m }{ m!}.
\end{equation}


\subsection{Taylori rida}

\paragraph{Taylori rida}
Iga funktsiooni võib punkti $x_0$ ümbruses esitada astmereana:
\begin{align}
f(x) & =f(x_0) + f'(x_0)(x-x_0) + \frac{1}{ 2}f''(x_0)(x-x_0)^2 +
  ... = \notag\\
  & =\sum^\infty_{i=0} f^{(i)}(x_0) \frac{(x-x_0)^i }{ i!}.
\end{align}
Tõestus: kirjuta samasugune
astmerida tundmatute kordajatega välja, võrruta $f(x-x_0)$-ga ja võta
järjest tuletisi. Taylori rea erijuht, kui $x_0=0$ on McLaureni rida:
\begin{equation}
f(x) =f(0) + f'(0)x + \frac{1}{ 2} f''(0)x^2 + ...
  = \sum^\infty_{i=0} f^{(i)}(0) \frac{x^i }{ i!}.
\end{equation}

\paragraph{Eksponendi astmerida}
\begin{equation}
 e^x = 1 + x + \frac{1}{ 2}x^2 + \frac{1}{ 3!}x^3 + ... = \sum^\infty_{i=0}
\frac{x^i }{ i!}.
\end{equation}
Tõestus: arenda $e^x$ Taylori ritta.

\paragraph{Eksponendi piirväärtus}
\begin{equation}
 \lim_{x\to 0} e^x = 1+x.
\end{equation}
Tõestus: eksponendi astmereast. Märkus: piir\-väärtus $1+x$ on kõige
tavalisem, mida on vaja kasutada. Olenevalt ülesandest tuleb arvestada
rohkem (või ka vähem) astmerea liikmeid.


\paragraph{Logaritmi astmerida}
\begin{equation}
 \ln x = (x-1) - \frac{(x-1)^2 }{ 2!} + \frac{(x-1)^3 }{ 3!} - \frac{(x-1)^4 }{
4!} + ...
\end{equation}
Tõestus: arenda Taylori ritta $x_0=1$ ümbruses.




\newpage
\subsection{Summade astmed}

\begin{align}
\left( \sum_{i=1}^N x_i \right)^2 = &
  \sum_{i=1}^N x_i^2 + \sum_{\substack{i;j=1 \\ i\ne j}}^N
  x_i x_j
  \label{eq:(sum x)^2}\\
\left( \sum_{i=1}^N x_i \right)^3 = &
  \sum_{i=1}^N x_i^3 + 3 \sum_{\substack{i;j=1 \\ i\ne j}}^N
  x_i x_j^2 + \sum_{\substack{i;j;k=1 \\ i\ne j; j\ne k; k \ne i}}^N
  x_i x_j x_k\\
\left( \sum_{i=1}^N x_i \right)^4 = &
  \sum_{i=1}^N x_i^4 +
  4\sum_{\substack{i;j=1 \\ i\ne j}}^N x_i x_j^3 +
  3\sum_{\substack{i;j=1 \\ i\ne j}}^N x_i^2 x_j^2 +
  6\sum_{\substack{i;j;k=1 \\ i\ne j; j\ne k; k \ne i}}^N
    x_i x_j x_k^2 + \notag\\
  + &
  \sum_{\substack{i;j;k;l=1\\ i;j;k;l\ne}}^N x_i x_j
    x_k x_l
\end{align}
Ühekordsetes summades on $N$ liiget, kahekordsetes $N(N-1)$,
kolmekordsetes $N(N-1)(N-2)$ ning neljakordses $N(N-1)(N-2)(N-3)$.

Tuletuskäik lähtub viimasel juhul niisugustest mõtetest:
\begin{itemize}
\item Kui komponentide indeksid ei tohi olla võrdsed, siis on järgmist
komponenti võimalik valida ühe võrra vähem
\item Esimesel liikmel võetakse sisse kõik komponendid, seega on
$C_0^4$ varianti.
\item Teisel liikmel on kaks komponenti ($x_i$ ja $x_j$), ühte
võetakse kolm korda, teist korra.  Seega tuleb valida üks, mis iga
kord välja jäetakse.  Seega $C_1^4$ võimalust.
\item Kolmandal liikmel valitakse mõlemad komponendid kahe kaupa.
Kokku on $C_2^4=6$ võimalust kahe kaupa valida, kuna aga pole vahet
kumb komponentidest on kumb, siis jääb järele pool nendest.
\item Neljandal liikmel on kolm komponenti, ruutliikme valimiseks on
$C_2^4=6$ võimalust.  Kuna teised liikmed on esimeses astmes, siis on
küll vahe, kumbad me välja valime.  Jääb 6.
\item Viimane, kõiki üks kord, $C_4^4=1$.
\end{itemize}

Kui $X \sim i.i.d$, siis
\begin{equation}
  \label{eq:E(sum x)^2}
  \E \left( \sum^N x_i \right)^2 =
  N \E X^2 + N(N - 1) (\E X)^2 =
  N^2 (\E X)^2 + N \var X
\end{equation}



\newpage
\subsection{Maatriksalgebra}

\begin{align}
 \mat{A}^{-1} = \left[
 \begin{array}{rr}
 a_{11} & a_{12}\\
 a_{21} & a_{22}
 \end{array}
 \right]^{-1} & =
 \frac{1}{|A|} \left[
 \begin{array}{rr}
  a_{22}        & -a_{21}\\
 -a_{12}        & a_{11}
 \end{array}
 \right] \\
%
\begin{bmatrix}
\mat{A}      & \mat{B} \\
\mat{C}      & \mat{D}
\end{bmatrix}^{-1} &=
\begin{bmatrix}
\mat{E}^{-1}                      & -\mat{E}^{-1}\mat{B}\mat{D}^{-1} \\
-\mat{D}^{-1}\mat{C}\mat{E}^{-1}  & \mat{F}^{-1}
\end{bmatrix}, \quad\mbox{kus} \\
&\mat{E}        = \mat{A} - \mat{B}\mat{D}^{-1}\mat{C} \notag\\
&\mat{E}^{-1}   = \mat{A}^{-1} + \mat{A}^{-1}\mat{B}\mat{F}^{-1} \mat{C} \mat{A}^{-1} \notag\\
&\mat{F}             = \mat{D} - \mat{C} \mat{A}^{-1} \mat{B} \notag\\
&\mat{F}^{-1}        = \mat{D}^{-1} \mat{C} \mat{E}^{-1} \mat{B} \mat{D}^{-1} \notag\\
\end{align}


\selectlanguage{english}
\subsection{Differentiation of Matrices and Vectors}

Definition: let $\lambda$ be a scalar, $\vec{x}$ and $\vec{\beta}$ $K \times 1$
vectors, and $\mat{A}$ a $K\times K$ matrix.  Define:
\begin{equation}
  \frac{\partial \lambda}{\partial \vec{x}}
  =
  \begin{bmatrix}
    \pderiv{x_1}\lambda\\
    \dots\\
    \pderiv{x_K}\lambda
  \end{bmatrix}
\end{equation}
More results:
\begin{alignat}{2}
  \frac{\partial \vec{x}}{\partial \vec{x}'} &= 
  \begin{bmatrix}
    \pderiv[x_{1}]{x_{1}} & \pderiv[x_{1}]{x_{2}} & \hdots & \pderiv[x_{1}]{x_{K}}\\[1.2ex]
    \pderiv[x_{2}]{x_{1}} & \pderiv[x_{2}]{x_{2}} & \hdots & \pderiv[x_{2}]{x_{K}}\\[1.2ex]
    \vdots               & \vdots              & \ddots & \vdots\\[1.2ex]
    \pderiv[x_{K}]{x_{1}} & \pderiv[x_{K}]{x_{2}} & \hdots & \pderiv[x_{K}]{x_{K}}\\
  \end{bmatrix}
  =
  \mat{I}
  &
  \frac{\partial \vec{x}'}{\partial \vec{x}} &= \mat{I}
  \\
  % 
  \frac{\partial \vec{x}' \vec{\beta}}{\partial \vec{x}} &= 
  \begin{bmatrix}
    \pderiv{x_{1}}( \beta_{1} x_{1} + \beta_{2} x_{2} + \hdots + \beta_{K} x_{K})\\[1.2ex]
    \pderiv{x_{2}}( \beta_{1} x_{1} + \beta_{2} x_{2} + \hdots + \beta_{K} x_{K})\\[1.2ex]
    \vdots\\
    \pderiv{x_{K}}( \beta_{1} x_{1} + \beta_{2} x_{2} + \hdots + \beta_{K} x_{K})\\[1.2ex]
  \end{bmatrix}
  =
  \vec{\beta}
  \\
  % 
  \pderiv[\mat{A} \vec{x}]{\vec{x}'} &= \mat{A}
  &
  \pderiv[\vec{x}'\mat{A}]{\vec{x}} &= \mat{A}
  \\
  % 
  \frac{\partial \vec{x}'\vec{x}}{\partial \vec{x}} &= \mat{I}\vec{x} + \vec{x}\mat{I} = 2\vec{x} 
  &
  \pderiv[\vec{x}' \mat{A} \vec{x}]{\vec{x}} 
  &= (\mat{A} + \mat{A}') \vec{x}
\end{alignat}
Let $\vec{f}(\vec{x})$ be a $M\times 1$ vector function and $\mat{B}$
a $M\times M$ matrix.
\begin{equation}
  \pderiv[\vec{f}(\vec{x})' \mat{B} \vec{f}(\vec{x})]{\vec{x}} 
  = 
  \left[ \vec{D}\,\vec{f}(\vec{x}) \right]'
  (\mat{B} + \mat{B}') \vec{f}(\vec{x}),
\end{equation}
where $\vec{D}\,\vec{f}(\vec{x})$ is Jacobian Matrix \eqref{sec:jacobian_matrix}.

\selectlanguage{estonian}
Oluline: ühte korrutamist ei tohi teiseks muuta. Näiteks kui avaldis
sisaldab nii maatrikskorrutist kui skalaariga korrutamist (skalaariga
korrutamine on põhimõtteliselt sama mis Kroneckeri korrutis
$\otimes$), ei tohi endist skalaariga korrutamist diferentseerimise
järel tõlgendada maatrikskorrutisena.  Mis siis et skalaari asemel on
nüüd maatriks:
\begin{equation}
\frac{\partial}{\partial \vec{\beta}'} \left[ ( \vec{\beta}'\vec{x} ) \otimes \vec{y}
  \right] =
  \frac{\partial \vec{\beta}'\vec{x}}{\partial \vec{\beta}'} \otimes \vec{y} =
  \vec{x}' \otimes \vec{y} = \vec{y}\vec{x}'.
\end{equation}
Skalaariga korrutamisel korrutatakse kõik maatriksi elemendid läbi
sama skalaariga, seega pääle tuletise võtmist tuleb kõik vektori $\vec{y}$
elemendid läbi korrutada tuletisvektoriga $\vec{x}'$.


\clearpage
\subsection{Võrratused}


\selectlanguage{english}
\subsubsection{Hölder's inequality}
Let $X$ and $Y$ be random variables.
\begin{equation}
  \E |XY| 
  \le 
  \left\{ \E \left[ |X|^\frac{1}{\alpha} \right] \right\}^\alpha 
  \left\{ \E \left[ |X|^\frac{1}{1 - \alpha} \right] \right\}^{1 -\alpha}
\end{equation}

\subsubsection{Jensen Inequality}

\begin{equation}
E f(x) < f( Ex),
\end{equation}
If $f(x)$ is concave.


\subsubsection{Triangle Inequality}
\begin{equation}
  |x + y| \le |x| + |y|
\end{equation}

\selectlanguage{estonian}
\subsubsection{Cauchy-Schwartzi võrratus}
\begin{equation}
  |<x,y>| \le \|x\| \cdot \|y\|
\end{equation}
\selectlanguage{english}
For sequences
\begin{equation}
\left( \sum a_i b_i \right)^2 \le
  \left( \sum a_i \right)^2 \left( \sum b_i \right)^2.
\end{equation}
For functions:
\begin{equation*}
  <x,y> = \int x \cdot y \, \dif x
  \qquad
  \| x \| = \sqrt{<x,x>}
\end{equation*}
\selectlanguage{estonian}
Vektorkujul:
\begin{equation}
(\vec a \cdot \vec b)^2 \le \|\vec a\|^2 \|\vec b\|^2
\end{equation}
ehk siis ka
\begin{equation}
\sum_i \vec z_i \vec z_i' \ge
  \frac{\sum_i a_i \vec z_i \sum_i a_i \vec z_i'}{\sum_i a_i^2}
\end{equation}


\selectlanguage{english}
\subsubsection{Inequalities, containing exponent}
Proof in most cases by analysing the corresponding function.
\begin{align}
  \me^a &\ge a\\
  %
  1 - \me^{-a} &< a\\
  %
  (1 - \me^{-a}) \me^{-a} &< a\\
  (1 + a) \me^a & \ge (1 + 2a)\\
  %
  (1 + a) \me^{-a} &< 1
  \qquad \text{if} \quad a > 0\\
  %
  (a - 1) \me^a &> -1
  \qquad \text{if} \quad a > 0\\
  %
  (1 + a^2) \me^{-a} &< 1
  \qquad \text{if} \quad a > 0\\
  %% e^a - a^b
  \me^{-a} - \me^{-b} &< -a + b 
  \qquad \text{if} \quad 0 < a < b\\
  %
  \me^a - \me^b 
  &= 
  (a-b) + \frac{1}{2}(a^2 - b^2) + \frac{1}{6}(a^3 - b^3) + \dots\\*
  &\gtrless (a - b)
  \qquad \text{if} \quad a \gtrless b\\
  % 
  a \me^{-b} - b\me^{-a}
  & \gtrless
  a - b
  \qquad \text{if} \quad a \gtrless b\\
  %
  a^2 \me^{b} - b^2 \me^{a}
  & \gtrless
  a^2 - b^2 +
  ab(a - b) +
  \frac{1}{6}a^2 b^2 (b - a) +
  \frac{1}{24}a^2 b^2 (b^2 - a^2) + \dots
  \notag\\*
  & \text{if} \quad b \gtrless a
\end{align}




\newpage
\section{Probability and Statistics}
\label{sec:probatility_and_statistics}

\subsection{Combinatorics}
\label{sec:combinatorics}

\paragraph{Combinations}

Number of combinations of selecting $k$ items out of $n$ where order
does not matter:
\begin{equation}
  \begin{pmatrix}
    n \\ k \\
  \end{pmatrix}
  \equiv
  C_{k}^{n}
  =
  \frac{n!}{k! (n-k)!}
\end{equation}

\subsection{Distributions: General Concepts}


\selectlanguage{estonian}

\label{sec:jaotused_mqisted}

\subsubsection{Stohhastiline domineerimine}
Olgu $F$ ja $G$ jaotusfunktsioonid.  $G$ esimest järku domineerib
$F$-i kui $G(x) \le F(x) \quad \forall x$ ehk $\bar G(x) \ge \bar F(x)
\forall x$.  Laias laastus annab $G$ suuremad väärtused kui $F$.

\subsubsection{Sõltumatud juhuslikud muutujad}
$X$ ja $Y$ on sõltumatud $\Leftrightarrow f(x,y) = f_X(x) f_Y(y)
\Leftrightarrow F(x,y) = F_X(x) F_Y(y)$.

\selectlanguage{english}
\subsubsection{Expectations}
Let support of random variable $X$ be $[a, b]$.  Expectation of $X$
\begin{align}
  \E X
  &=
  \int_a^b x \, \dif F_X(x)\\
  &=
  a + \int_a^b \bar F_X(x) \,\dif x.
\end{align}

\paragraph{Law of iterated expectations}
\begin{equation}
  \E_X
  \left[ \E [Y|X] \right]
  =
  \E [Y]
\end{equation}


\subsubsection{Characteristic  Function}
\label{sec:characteristic_function}

Let $X$ be a random variable.  It's characteristic function:
\begin{equation}
  \label{eq:characteristic_function}
  \phi_{X}(t) = \E \me^{itX}
\end{equation}

Properties: for independent random variables $X_{1}$, $X_{2}$
\begin{equation}
  \phi_{X_{1} + X_{2}}(t) = \phi_{X_{1}}(t) \cdot \phi_{X_{2}}(t)
\end{equation}

\selectlanguage{estonian}
\subsubsection{Momendifunktsioon (\emph{moment generating function})}

MGF avaldub ühemõõtmelisel juhul:
\begin{equation}
M_x (s) = \E e^{sx} = \int e^{sx} f(x) dx.
\end{equation}
Momendifunktsiooni omadused:
\begin{equation}
M_x'(0) = \E x \qquad M_x''(0) = \E x^2 \qquad 
  M_x^{(n)}(0) = \E x^n.
\end{equation}
Kasulik asi on ka $\log x$-i momendifunktsioon:
\begin{equation}
M_{\log x}(s) = \E \me^{s \log x} = \E x^s.
\end{equation}

$N$-mõõtmelisel juhul avaldub MGF:
\begin{multline}
M( s_1,s_2,\ldots,s_N) = \E e^{\sum^N s_i x_i} = \\
= \idotsint e^{s_1 x_1}
  e^{s_2 x_2} \ldots e^{s_N x_N} f(x_1,x_2,\ldots, x_N) dx_1 dx_2 \ldots
  dx_N. 
\end{multline}

\subsubsection{Kumulandifunktsoon (\emph{cumulant-generating function})}

KGF avaldub MGF-i kaudu:
\begin{equation}
K_x(s) = \log M(s) 
  \qquad \mbox{või} \qquad
K(s_1,s_2,\ldots,s_N) = \log M( s_1,s_2,\ldots,s_N).
\end{equation}
KGF-i omadus (ühemõõtmelisel juhul):
\begin{eqnarray}
K_x'(0) = \E x \\
K_x''(0) = \var x\\
K_x'''(0) = \E(x-\E x)^3
\end{eqnarray}
ja kahemõõtmelisel juhul:
\begin{equation}
\frac{\partial^2 K(0,0)}{\partial s_1 \partial s_2} =
  \cov( x_1, x_2).
\end{equation}

\selectlanguage{english}

\subsubsection{Probability Generating Function}
\label{sec:probability_generating_function}

For discrete, non-negative random variables
\begin{equation}
  G(z) = \E (z^X) = \sum_{x=0}^{\infty}p(x)z^x.
\end{equation}
Properties:
\begin{align}
  p(k) &= \operatorname{Pr}(X = k) = \frac{G^{(k)}(0)}{k!}
\end{align}

\subsubsection{Distribution Function of a Function of Random Variable}
\selectlanguage{estonian}
Olgu juhuslikud muutujad $X$ ja $Y$ kusjuures $X=X(Y)$.  Siis
\begin{equation}
  F_x(x) = \Pr[X < x] = \Pr[X < x(y)] = F_x[x(y)] = F_y(y)
\end{equation}
ja
\begin{equation}
  f_y(y) = F_y'(y) = \frac{\dif}{\dif y} F_x[x(y)] =
  \frac{\dif}{\dif x} F_x(x) \frac{\dif x}{\dif y} = f_x[x(y)]
  \frac{\dif x}{\dif y}. 
\end{equation}


\subsection{Ühemõõtmelised diskreetsed jaotused}

\subsubsection{Bernoulli jaotus}
Kõige lihtsam kulli-ja-kirja jaotus.  Olgu sündmuse $A$ tõenäosus
$p$.  Siis juhuslik muutuja
\begin{equation}
  Y = 
  \begin{cases}
    1 & \text{kui $A$},\\
    0 & \text{kui $\bar A$.}
  \end{cases}
\end{equation}
on Bernoulli jaotusega, kusjuures $\E Y = p$ ja $\var Y = p(1-p)$.

\subsubsection{Binoomjaotus}
On $N$ ühesuguse sõltumatu Bernoulli jaotusega juhusliku muutuja summa
jaotus.  Olgu $X = \sum^N Y_i$ kus $Y_i$ on Bernoulli jaotusega
parameetriga $p$.  Siis
\begin{eqnarray}
\Pr(X = x) &=& \binom{N}{x} p^x (1-p)^{N-x} \quad x\in \{0, \ldots ,N\}\\
E Y &=& Np \\
E ( Y - EY)^2 &=& N p( 1 - p) \\
E ( Y - EY)^3 &=& N p( 1 - p)( 1 - 2p)\\
E ( Y - EY)^4 &=& N p( 1 - p)( -1 + 3p - 3p^2)
\end{eqnarray}
Tõestus: ühe katse korral kirjuta lahti $(y-p)^n$, arvesta et $Ey^n=p$
($i\not=0$).  N sõltumatu katse korral momendid liituvad.

\subsubsection{Diskreetne jaotus}
Jaotus kus juhslikul muutujal võib olla lõplik hulk diskreetseid
väärtusi. 

\subsubsection{Geomeetriline jaotus ($Geo(p)$)}
Geomeetriline jaotus kirjeldab mingi hulga Bernoulli jaotusega
suuruste järjest esinemist.  Olgu sündmuse tõenäosus $p$.  Tõenäosus,
et järjest toimub $n$ sündmust ja seejärel sündmuste jada katkeb on:
\begin{equation}
f(n) = (1-p)p^n.
\end{equation}
Jaotuse omadused:
\begin{eqnarray}
\E n = \frac{1-p}{p}\\
\var n = \frac{1-p}{p^2}
\end{eqnarray}


\subsubsection{Multinoomjaotus}
Olgu üksikul katsel $M$ võimalikku tulemust $A_1 \ldots A_M$ vastavate
tõenäosustega $p_1 \ldots p_M$, kusjuures $\sum^M p_i = 1$.  Olgu
$N$-katselises seerias $N_i$  realiseerunud sündmuste $A_i$ arv.
Siis:
\begin{align}
E N_i &= Np_i \\
\var N_i &= N p_i ( 1 - p_i) \\
&\ldots \notag\\
\cov( N_i, N_j) &= -N p_i p_j
\end{align}
Tõestus: momentide arvutamisel võib multinoomjaotuse taandada
binoomjaotuseks, kovariatsiooni jaoks kirjuta definitsioon lahti,
arvesta et $E A_i A_j = 0$.


\subsubsection{Poissoni jaotus}
Poissoni jaotusega on sõltumatute sündmuste arv ajaühikus.  Kui
ajaühikus toimub keskmiselt $\lambda$ sündmust, siis tõenäosus, et
toimub $n$ sündmust aja $t$ jooksul on:
\begin{eqnarray}
f(n)\equiv p_p(n) &=& \frac{\me^{-\lambda t}(\lambda t)^n}{n!}\\
                                %
P_p(n) &\equiv& \sum_{s=1}^n p_p(s)
\end{eqnarray}
Jaotuse omadused: jaotus on log-kumer,
\begin{equation}
  \E n = \lambda t \qquad \var n = \lambda t
\end{equation}
Tõenäosus, et mingi aja $t$ jooksul ei toimu ühtegi sündmust, $f(0)$, on
eksponentjaotusega $\mathcal{E}(\lambda)$.

ML-hinnang: Kui vaatluse $i$ jooksul, mille kestus on $t_i$ 
toimub $n_i$ sündmust, siis kõigi vaatluste ML hinnang on:
\begin{equation}
  \hat\lambda = \frac{\sum n_i}{\sum t_i} 
  \qquad \mbox{ja} \qquad
  \var\hat\lambda = \frac{\sum n_i}{\left( \sum t_i \right)^2}
\end{equation}

Poissoni summa tuletis aja järgi: Olgu
\begin{equation}
  \vartheta(t) =
  \sum_{s=0}^S Q(s) \frac{(\lambda t)^2}{s!} \me^{-\lambda t} = 
  \E_s Q(s)
\end{equation}
siis
\begin{eqnarray}
  \pderiv{t} \vartheta(t) &=&
  \lambda \sum_{s=0}^{S - 1}
  \left[ Q(s + 1) - Q(s) \right] p_p(s) -
  \lambda Q(S) p_p(S) =\\
                                %
  &=&
  \lambda \sum_{s=1}^S Q(s) 
  \left[ p_p(s - 1) - p(s) \right]
  - \lambda Q(0) p_p(0)\\
                                %
  \pderiv{t} P_p(s) &=& -\lambda p_p(s)\\
                                %
  \pderiv{t} p_p(s) &=&
  \begin{cases}
    -\lambda p_p(s) & \text{kui $s$ = 0}\\
    -\lambda p_p(s) + \lambda p_p(s-1) & \text{kui $s$ > 0}\\
  \end{cases}
\end{eqnarray}

\selectlanguage{english}

\subsubsection{Skellam Distribution $PD(\lambda,\delta)$}
\label{sec:skellam_distribution}

Skellam distribution is distribution of difference of two independent
Poisson RV-s.  Let $N = X - Y$ where $X \sim Poisson(\lambda)$ and $Y
\sim Poisson(\delta)$:
\begin{align}
  P(N=n) &= 
  \me^{-(\lambda + \delta)}
  \left(\frac{\lambda}{\delta} \right)^{\frac{n}{2}}
  I_{n}(2\sqrt{\lambda\delta})
  \\
  \E N &= \lambda - \delta
  \\
  \var N &= \lambda + \delta
\end{align}



\newpage
\subsection{1D Continuous Distributions}
\selectlanguage{estonian}

\subsubsection{Eksponentjaotus $\mathcal{E}(\theta)$}
Eksponentjaotus kirjeldab konstantse kiirusega hääbuvaid protsesse.
Tõenäosustihedus:
\begin{equation}
f(t) = \theta e^{-\theta t}, \qquad t \ge 0, \theta > 0
\end{equation}
jaotusfunktsioon:
\begin{equation}
F(t) = 1 - e^{-\theta t},
\end{equation}
momendifunktsioon:
\begin{equation}
M_T (s) = \frac{1}{1-\frac{s}{\theta}}, \quad s<\theta.
\end{equation}
Momendid:
\begin{eqnarray}
\E T &=& \frac{1}{\theta}\\
\E T^2 &=& \frac{2}{\theta^2}\\
\var(T - \E(T))^2 &=& \frac{1}{\theta^2}.
\end{eqnarray}

$\log T$ on esimest liiki ekstreemväärtuste jaotusega. $\log T$-ga
seotud suurused on:
\begin{eqnarray}
M_{\log T} (s) &=& \frac{\Gamma(s+1)}{\theta^s}\\
K_{\log T} (s) &=& \log \Gamma(s+1) - s \log \theta\\
\E \log T &=& \psi(1) - \log \theta\\
\var \log T &=& \psi'(1),
\end{eqnarray}
kus $\psi$ on digamma funktsioon.

Kui $z_1 \sim \mathcal{E}(\theta_1)$ ja $z_2 \sim
\mathcal{E}(\theta_2)$ siis
\begin{equation}
  \log z_1 - \log z_2 \sim \frac{\theta_1}{\theta_1 + \theta_2
  \me^{-x}} \sim \Lambda(x), \qquad \mbox{kui} \quad \theta_1 = \theta_2
\end{equation}
Tõestus: arvesta et $\Pr(z_1/z_2 < \alpha) = \Pr(z_1 < \alpha z_2)$ ja
integreeri. 


\subsubsection{F-jaotus $F(n_1, n_2)$}
F-jaotus tekib kahe $\chi^2$ jaotusega suuruse jagamisel.  Kui $w_1
\sim \chi_{n_1}^2$ ja $w_2 \sim \chi_{n_2}^2$ siis
\begin{equation}
\frac{\frac{w_1}{n_1}} {\frac{w_2}{n_2}} \sim F( n_1, n_2).
\end{equation}
Tihedusfunktsioon:
\begin{equation}
f(x) = \frac{
  \left(\frac{n_1}{n_2}\right)^\frac{n_1}{2} x^{\frac{n_1}{2}-1}}
  {B\left( \frac{n_1}{2}, \frac{n_2}{2} \right)
    \left( 1 + \frac{n_1}{n_2}x\right)^\frac{n_1+n_2}{2} }
\end{equation}


\subsubsection{Gammajaotus $\mathcal{G(\alpha,\beta)}$}
Kasutatakse heterogeensuse kirjeldamiseks kestusmudelites.
Tõenäosustihedus:
\begin{equation}
f(x) = \frac{1}{\beta^\alpha} \frac{1}{\Gamma( \alpha)} x^{\alpha-1}
  e^{-\frac{x}{\beta}} \qquad x > 0,
\end{equation}
$\alpha >0$ kirjeldab kuju ja $\beta>0$ on skaalaparameeter.  Märkus:
mõnikord kasutatakse ka parameetreid $(\alpha,\frac{1}{\beta})$.

Momendifunktsioon ja ootused:
\begin{eqnarray}
M_x (s) &=& \frac{1}{(\beta s - 1)^\alpha}\\
K_x (s) &=& -\alpha \log(\beta s - 1)\\
\E x &=& \beta \alpha \\
\E x^2 &=& \beta^2 \alpha( \alpha + 1) \\
\var x &=& \beta^2 \alpha.
\end{eqnarray}

$\log x$ kirjeldav momendifunktsioon ja ootused:
\begin{eqnarray}
M_{\log x} (s) &=& \frac{\Gamma(s + \alpha)}{\Gamma(\alpha)}\beta^s\\
K_{\log x} (s) &=& s\log \beta + \log \Gamma(s + \alpha) -
  \log\Gamma(\alpha)\\
\E\log x &=& \log\beta + \psi(\alpha)\\
\var\log x &=& \psi'(\alpha)
\end{eqnarray}

Erijuht on \emph{normaalne gammajaotus}, mille keskväärtus on 1.  Sel
juhul 
\begin{equation}
\alpha = \frac{1}{\beta} \equiv \eta
\end{equation}
ja jaotusfunktsioon
\begin{equation}
f_x(x) = \eta^\eta \frac{1}{\Gamma(\eta)} x^{\eta - 1} \me^{-\eta x}.
\end{equation}
Sel juhul:
\begin{eqnarray}
  \E x &=& 1\\
  \E x^2 &=& 1 + \frac{1}{\eta}\\
  \var x &=& \frac{1}{\eta}
\end{eqnarray}

Teine oluline erijuht on $\chi^2$-jaotus.  Kui $\alpha=\frac{k}{2}$ ja
$\beta=2$, siis $X$ jaotusfunktsioon on
\begin{equation}
f_x (x) = \frac{1}{2^\frac{k}{2}} \frac{1}{\Gamma(\frac{k}{2})}
  y^{\frac{k}{2} - 1} \me^{-\frac{y}{2}}.
\end{equation}
Tolle kohta öeldakse $\chi^2(k)$ jaotus.


\subsubsection{Hii-ruut jaotus $\chi^2(k)$}
$\chi^2 (k)$ jaotus tekib kui liita kokku $k$ normaaljaotusega
juhusliku suuruse ruutu.  Jaotusfunktsioon:
\begin{equation}
f_x (x) = \frac{1}{2^\frac{k}{2}} \frac{1}{\Gamma(\frac{k}{2})}
  y^{\frac{k}{2} - 1} \me^{-\frac{y}{2}}.
\end{equation}


\subsubsection{Log-normaalne jaotus $LN(\mu,\sigma^2)$}
Log-normaalse jaotusega on juhuslik suurus, mille logaritm on
normaaljaotusega.  Tihedusfunktsioon:
\begin{equation}
f_x (x) = \frac{1}{\sqrt{2\mpi}\sigma x}
  \me^{-\frac{1}{2}\left[\frac{(\log x - \mu)}{\sigma}\right]^2}.
\end{equation}
Omadused:
\begin{eqnarray}
\E x &=& \me^{\mu + \frac{1}{2}\sigma^2}\\
\var x &=& \me^{2\mu + \sigma^2} (\me^{\sigma^2} - 1).
\end{eqnarray}

\subsubsection{Log-ühtlane jaotus}
Kasutatakse palgajaotuse kirjeldamiseks.  Tihedusfunktsioon:
\begin{equation}
  f(x) = \frac{1}{x}\frac{1}{\log \beta - \log \alpha}, \qquad
  0 \le \alpha \le x \le \beta < \infty.
\end{equation}

\selectlanguage{english}
\subsubsection{Logistic Distribution}
\begin{align}
  \text{cdf}\qquad & \Lambda(x) = \frac{e^x}{1+e^x} = \frac{1}{1 + \me^{-x}}
  \\
  \text{pdf}\qquad & f(x) = \frac{\me^x}{(1 + \me^x)^2} = \frac{\me^{-x}}
  {(1 + \me^{-x})^2}
  \\
  & f'(x) = \me^{-x} \frac{\me^{-x} - 1}{(\me^{-x} + 1)^3}
  \\
  \text{MGF}\qquad &
  M(s) = \int \frac{\me^{x(s+1)}}{(1 + \me^x)^2} \dif x
\end{align}
Logistic distribution is symmetric around 0, i.e. $\Lambda(x) = 1 -
\Lambda(-x)$ and $f(x) = f(-x)$.

\subsubsection{Normal Distribution $N(\mu,\sigma^2)$}
Sum of many independent random disturbations tends to be normally
distributed (Central Limit Theorem.)
\begin{align}
  F(x; \mu, \sigma) \equiv & 
  \Phi\left( \frac{x - \mu}{\sigma} \right)
  &
  \text{cannot be expressed analytically}
  \\
  f(x; \mu, \sigma) \equiv & 
  \frac{1}{\sigma} \phi\left( \frac{x - \mu}{\sigma} \right)
  &=
  \frac{1}{\sqrt{2 \mpi}}\frac{1}{\sigma}
  \me^{\scriptstyle\frac{1}{2}\frac{(x - \mu)^2}{\sigma^2}}
\end{align}
Characteristic function:
\begin{equation}
  \phi_{X}(t) = \me^{i t \mu - \frac{1}{2}\sigma^2 t^2}
\end{equation}
Moment generating function
\begin{equation}
M_x (s) = \me^{\mu s + \frac{1}{2}{\sigma^2 s^2}}
\end{equation}

Properties: if $X_{i} \sim N(\mu_{i}, \sigma_{i}^{2})$ are
independent normals
\begin{equation}
  \sum_{i} X_{i} \sim 
  N \left(\sum_{i} \mu_{i}, \sum_{i} \sigma_{i}^{2} \right).
\end{equation}

\paragraph{Conditional Expectations}
If $X \sim N(\mu, \sigma)$, 
\begin{eqnarray}
\E [X|X < a ]
&=&
 \mu - \sigma \frac {\phi( \frac{a - \mu}{\sigma} )} {\Phi(
   \frac{a - \mu}{\sigma})}
 =
 \mu - \sigma\lambda \left( \frac{a - \mu}{\sigma} \right)
\\
\E [X|X>a ]
&=&
\mu + \sigma \frac {\phi( \frac{a - \mu}{\sigma} )} 
{1 - \Phi(\frac{a - \mu}{\sigma})}
=
\mu + \sigma \lambda \left( \frac{ \mu - a}{\sigma} \right)
\\
\E [X|X \in [a,b]]
&=&
 \mu - \sigma \frac
 {\phi \left( \frac{b - \mu}{\sigma} \right) - 
   \phi\left( \frac{a - \mu}{\sigma}\right)}
 {\Phi\left(\frac{b - \mu}{\sigma}\right) - 
   \Phi\left(\frac{a - \mu}{\sigma}\right)}
\end{eqnarray}

If $X \sim N(0, \sigma)$, 
\begin{align}
\E[X|X<a]
 &=
 -\sigma \lambda(\frac{a}{\sigma})
\\
\E[X|X>a] 
&=
\sigma \lambda(-\frac{a}{\sigma})
\\
\E[X^2|X < a]
&=
\sigma^2 - \sigma a \lambda \left(\frac{a}{\sigma} \right)
\\
\E[X^2|X > a]
&=
\sigma^2 + \sigma a \lambda \left(-\frac{a}{\sigma} \right)
\\
\E[X^2|X > -a \land X < a]
&=
\sigma^2 - 
2 \frac{\sigma a \phi\left(\frac{a}{\sigma} \right)}
{1 - 2 \Phi\left(\frac{a}{\sigma} \right)}
\\
\E[X^2|X < -a \lor X > a] 
&=
\E[X^2|X > a]
\\
\var[X|X <a] 
&= 
\sigma^2 \left[1 - 
  \frac{a}{\sigma} \lambda\left(\frac{a}{\sigma}\right) -
  \lambda^2(\frac{a}{\sigma}) \right]
\\
\var[X|X > a] 
&=
\sigma^2 \left[1 + 
  \frac{a}{\sigma} \lambda \left(-\frac{a}{\sigma} \right) - 
  \lambda^2(-\frac{a}{\sigma}) \right] \\
\end{align}

Let $X \sim N(\mu_{X}, \sigma_{X}^{2})$ and $Y \sim N(\mu_{Y},
\sigma_{Y}^{2})$, $X \independent Y$.  Now
\begin{align}
  \E[X|X < Y] 
  &=
  \mu_{X}
  - \frac{\sigma_{X}^{2}}{\sqrt{\sigma_{X}^{2} + \sigma_{Y}^{2}}}
  \lambda\left( -\frac{\mu_{X} - \mu_{Y}}
    {\sqrt{\sigma_{X}^{2} + \sigma_{Y}^{2}}}
    \right)
  \\
  \E[X|X > Y] 
  &=
  \mu_{X}
  + \frac{\sigma_{X}^{2}}{\sqrt{\sigma_{X}^{2} + \sigma_{Y}^{2}}}
  \lambda\left( \frac{\mu_{X} - \mu_{Y}}
    {\sqrt{\sigma_{X}^{2} + \sigma_{Y}^{2}}}
    \right).
\end{align}
Proof: write $\E[X|X < Y] = \E[X|Z < 0]$ where $Z = X-Y$.  Now
follows from \eqref{eq:normal2d_E[X1|X2<a]}

\subsubsection{Pareto Distribution}
Describes the upper part of many highly unequal distributions. 
\begin{align}
F_X (x) 
&= 
1 - \left( \frac{x_0}{x} \right)^\alpha, 
\quad x \ge x_0 > 0; \quad \alpha > 0
\\
f_{X}(x) &=
\alpha \, x_{0}^{\alpha} \, x^{-\alpha-1}
\\
\E X &=
\frac{\alpha}{\alpha - 1} x_{0},
\quad \alpha > 1
\end{align}
Properties:
\begin{itemize}
\item power law:
  \begin{math}
    \log f_{X}(x) 
    = 
    log \left( \alpha \, x_{0}^{\alpha} \right)
    -(\alpha + 1) \log x
  \end{math}
  is linear on log-log scale.
\item it is \emph{scale-free}: there is no features in the right tail,
  wherever you look, you have most observations that are smaller, but
  you also have observations that are way larger.
\end{itemize}


\selectlanguage{estonian}
\subsubsection{Pööratud normaaljaotus}
Tihedusfunktsioon:
\begin{equation}
f(t) = \frac{1}{t^\frac{3}{2}}
  \phi \left( \frac{ \mu t - 1}{\sigma\sqrt{t}} \right)
\end{equation}
ja jaotusfunktsioon:
\begin{equation}
F(t) = 
  \Phi \left( \frac{\mu t -1}{\sigma\sqrt{t}} \right) -
  e^{2\frac{\mu}{\sigma^2}}
    \Phi \left( - \frac{\mu t+1}{\sigma \sqrt{t}} \right).
\end{equation}
Kõik momendid on olemas kui $\mu>0$:
\begin{eqnarray}
\E T &=& \frac{1}{\mu} \\
\var T &=& \frac{\sigma^2}{\mu^3}.
\end{eqnarray}
Kui $\mu=0$, on jaotus korralik, positiivsed momendid aga puuduvad.  


\subsubsection{$t$-Distribution}
\label{sec:t-distribution}

Used for \emph{$t$-test}.  For $n$ \emph{degrees of freedom}:
\begin{align}
  f(x)
  &=
  \frac{\Gamma \left( \frac{n + 1}{2} \right)}
  {\sqrt{n \mpi} \Gamma \left( \frac{n}{w} \right)}
  \left(
    1 + \frac{x^{2}}{n}
  \right)^{-\frac{n + 1}{2}}
  \\
  \E X &= 0
  \\
  \var X &= \frac{n}{n - 2}
  \\
  \text{skewness } g_{1} &= 0
  \\
  \text{curtosis } g_{2} &= \frac{3n - 6}{n - 4} \qquad (n > 4)
\end{align}

\subsubsection{Triangular Distribution}

\begin{align}
  F(x) &= 
         \begin{cases}
           0 & x < 0\\
           x^{2} & 0 \le x \le 1\\
           1 & x > 1
         \end{cases}
  \\               
  f(x) &= 
         \begin{cases}
           2x & 0 \le x \le 1\\
           0 & \text{elsewhere}
         \end{cases}
  \\
  \E X &= \frac{2}{3}
  \\
  \E X^{2} &= \frac{1}{2}
  \\
  \var X &= \frac{1}{18}
\end{align}

\subsubsection{Type-1 Extreme Value Distribution $EV_{1}$}
(Also Gumbel distribution.)
Describes $\log T$ when $T \sim \mathcal{E}(1)$.
\begin{align}
f(x) &= \me^{-x} \me^{-\me^{-x}}
\\
F(x) &=  \me^{-\me^{-x}}.
\\
\E x &= -c \approx 0,5772
\end{align}

\subsubsection{Type-2 Extreme Value Distribution $EV_{2}$}
(Also Fréchet distribution or inverse Weibull distribution.)
\begin{align}
  f(x) &= \alpha x^{-1-\alpha} \me^{-x^{-\alpha}}
  \\
  F(x) &= \me^{-x^{-\alpha}}
\end{align}


\subsubsection{Uniform Distribution $Unif(a,b)$}
\begin{align}
  \text{expectation}\quad
  \mu \equiv \E X &= \frac{1}{2}(a + b)
  \\
  \text{variance}\quad
  \sigma^{2} \equiv \E (X - \mu)^{2} & = \frac{1}{12}(b - a)
  \\
  \text{4th central moment}\quad
  m_{4} \equiv \E (X - \mu)^{4} &= \frac{(b - a)^{4}}{80}
\end{align}

\selectlanguage{estonian}
\subsubsection{Weibulli jaotus}
Weibulli jaotus on eksponentjaotuse üldistus, kasutatakse ajas
ühtlaselt kahaneva hasardi kirjeldamiseks.  Omadused:
\begin{eqnarray}
  F(t) &=& 1 - \me^{-(\lambda t)^\alpha}
  \\
  f(t) &=& \alpha \lambda^\alpha t^{\alpha - 1}
  \me^{-(\lambda t)^\alpha}\\
                                %
  \theta(t) &=& \alpha \lambda^\alpha t^{\alpha - 1}
\end{eqnarray}
\selectlanguage{english}
where $\lambda$ is scale- and $\alpha$ is the shape parameter.


\clearpage
\subsection{Multivariate Continuous Distributions}

\subsubsection{Multivariate Normal $N(\mu, \mat{\Sigma})$}

\selectlanguage{estonian}
N-mõõtmelise normaajaotuse jaotusfunktsioon: Olgu
\begin{equation}
  \vec{X} \sim N(\vec{\mu}, \mat{\Sigma}),
\end{equation}
kus $\vec\mu$ on keskväärtus ja $\mat\Sigma$ dispersioonimaatriks.
Siis:
\begin{equation}
f_{\vec{X}} (\vec x) = 
  (2\mpi)^{-\frac{n}{2}} 
  |\mat\Sigma|^{-\frac{1}{2}}
  \me^{-\frac{1}{2}(\vec x - \vec\mu)' \mat\Sigma^{-1} (\vec x -
    \vec\mu)}.
\label{eq:N-Dimensional_normal}
\end{equation}

\paragraph{2D Conditional Distributions}
Let
\begin{math}
  \vec{X}=
  \begin{pmatrix}
    X_1\\
    X_2
  \end{pmatrix}
  \sim
  N \left( 
    \begin{pmatrix}
      0\\0
    \end{pmatrix},
    \begin{pmatrix}
      \sigma_1^2 & \sigma_{12} \\ 
      \sigma_{12} & \sigma_2^2
    \end{pmatrix}
\right)
\end{math}.

The exponent in the distribution function can be expressed as
\begin{equation}
  -\frac{1}{2} \left(
    \frac{
      \sigma_{2}x_{1}^{2} - 2 \sigma_{12} x_{1} x_{2} +
      \sigma_{1}^{2} x_{2}^{2}}
    {\sigma_{1}^{2} \sigma_{2}^{2} - \sigma_{12}^{2}}
  \right)
  =
  -\frac{1}{2} \left[
    \frac{\left(
        x_{1} - 
        \displaystyle\frac{\sigma_{12}}{\sigma_{2}^{2}}x_{2}
      \right)^{2}}
    {\sigma_{1}^{2} - \displaystyle\frac{\sigma_{12}^{2}}{\sigma_{2}^{2}}}
    +
    \frac{x_{2}^{2}}{\sigma_{2}^{2}}
  \right].
  \label{eq:normal2D_exponent}
\end{equation}
Accordingly, based on Bayesian law the probability density of 2D normal
\begin{math}
f_{X_{1},X_{2}}(x_{1},x_{2}) = 
f_{X_{1}|X_{2}}(x_{1},x_{2}) f_{X_{2}}(x_{2}) =
f_{X_{2}|X_{1}}(x_{1},x_{2}) f_{X_{1}}(x_{1})
\end{math}
where all the conditional and marginal distribution functions are
normal:
\begin{align}
  (X_{1}|X_{2} = x_{2}) 
  &\sim
  N \left(
    \frac{\sigma_{12}}{\sigma_{2}^{2}} x_{2},
    \sigma_{1}^{2} - \frac{\sigma_{12}^{2}}{\sigma_{2}^{2}}
  \right)
  &
  X_{2} &\sim N(0, \sigma_{2}^{2})
  \\
  (X_{2}|X_{1} = x_{1}) 
  &\sim
  N \left(
    \frac{\sigma_{12}}{\sigma_{1}^{2}} x_{1},
    \sigma_{2}^{2} - \frac{\sigma_{12}^{2}}{\sigma_{1}^{2}}
  \right)
  &
  X_{1} &\sim N(0, \sigma_{1}^{2})
\end{align}

Distribution for $(X_{1}|X_{2} < a)$:
\begin{equation}
  f_{X1|X2 < a}(x_{1}) = 
  \frac{1}{\sigma_{1}}
  \frac{\phi \left( \frac{x_{1}}{\sigma_{1}} \right)}
  {\Phi \left( \frac{a}{\sigma_{2}} \right)}
  \Phi \left(
    \frac{a - \frac{\sigma_{12}}{\sigma_{1}^{2}} x_{1}}
    {\sqrt{\sigma_{2}^{2} - \frac{\sigma_{12}^{2}}{\sigma_{1}^{2}}}}
  \right)
\end{equation}

Distribution for $(X_{1}|X_{2} > a)$:
\begin{equation}
  f_{X1|X2 > a}(x_{1}) = 
  \frac{1}{\sigma_{1}}
  \frac{\phi \left( \frac{x_{1}}{\sigma_{1}} \right)}
  {\Phi \left( -\frac{a}{\sigma_{2}} \right)}
  \Phi \left(
    - \frac{a - \frac{\sigma_{12}}{\sigma_{1}^{2}} x_{1}}
    {\sqrt{\sigma_{2}^{2} - \frac{\sigma_{12}^{2}}{\sigma_{1}^{2}}}}
  \right)
\end{equation}

\paragraph{Conditional Expectations}
Let
\begin{math}
  \vec{X}=
  \begin{bmatrix}
    X_1\\
    X_2
  \end{bmatrix}
  \sim
  N(\vec{\mu}, \vec{\Sigma})
\end{math}, where
\begin{math}
\vec{\Sigma} = \left[
 \begin{array}{rr} \sigma_1^2 & \sigma_{12}\\ \sigma_{12} & \sigma_2^2
 \end{array}\right]
\end{math} and
\begin{math}
\vec{\mu}=\left[\begin{array}{r} \mu_1\\
 \mu_2\end{array}\right]
\end{math}:

\begin{equation}
  (X_1|X_2 = x_2)
  \sim 
  N\left(\mu_1 + \frac{\sigma_{12}}{\sigma_2^2}(x_2 - \mu_2),
    \sigma_1^2 - \frac{\sigma_{12}^2}{\sigma_2^2}\right)
\label{eq:tinglik normaaljaotus}
\end{equation}
(follows from \ref{eq:normal2D_exponent}) and
\begin{align}
  \E[X_1|X_2 < a] 
  & = 
  \mu_1 - \frac{\sigma_{12}}{\sigma_2}
  \lambda 
  \left( \frac{a - \mu_2}{\sigma_2} \right)
  \label{eq:normal2d_E[X1|X2<a]}
  \\
  \E[X_1|X_2 > a] 
  & = 
  \mu_1 + \frac{\sigma_{12}}{\sigma_2}
  \lambda \left( \frac{\mu_2 - a}{\sigma_2} \right)
  \\
  \E[X_{1}^{2}|X_{2} < a]
  &=
  \sigma_{1}^{2} - \frac{\sigma_{12}^{2}}{\sigma_{2}^{3}}
  a \lambda \left( \frac{a}{\sigma_{2}} \right)
  \\
  \E[X_{1}^{2}|X_{2} > a]
  &=
  \sigma_{1}^{2} + \frac{\sigma_{12}^{2}}{\sigma_{2}^{3}}
  a \lambda \left( -\frac{a}{\sigma_{2}} \right)
  \\
  \E[X_1^2|X_2 \in \mathcal{A}]
  &=
  \frac{\sigma_{12}^2}{\sigma_2^4}
  \E[X_2^2|X_2 \in \mathcal{A}] +
  \sigma_1^2 - \frac{\sigma_{12}^2}{\sigma_2^2}
  \\
  \var[ X_1|X_2 < a]
  & =
  \sigma_{1}^{2} 
  -
  \frac{\sigma_{12}^2}{\sigma_{2}^{3}} a 
  \lambda \left( \frac{a}{\sigma_{2}} \right) 
  - 
  \frac{\sigma_{12}^2}{\sigma_{2}^{2}}  
  \lambda^2 \left( \frac{a}{\sigma_{2}} \right)
  \label{eq:normal_var[X1|X2<a]}
  \\
  \var[ X_1|X_2 > a]
  & =
  \sigma_{1}^{2} 
  +
  \frac{\sigma_{12}^2}{\sigma_{2}^{3}} a 
  \lambda \left(-\frac{a}{\sigma_{2}} \right) 
  - 
  \frac{\sigma_{12}^2}{\sigma_{2}^{2}}  
  \lambda^2 \left(-\frac{a}{\sigma_{2}} \right)
  \label{eq:normal_var[X1|X2>a]}
\end{align}
Proof: write (\ref{eq:tinglik normaaljaotus}) $\Rightarrow X_1 =
\mu_1 + \frac{\sigma_{12}}{\sigma_2}(X_2 - \mu_2) + E$,
where $E$ and $X_2$ are independent (property of normal distribution).  Find $X_1|X_2
\in \mathcal{A} = \frac{\sigma_{12}}{\sigma_2^2} \E[X_2|X_2 \in
\mathcal{A}] + E$.


\paragraph{Multiplying normals}

\begin{equation}
\frac{1}{\sigma_1}\phi\left( \frac{x-ay}{\sigma_1} \right)
  \frac{1}{\sigma_2}\phi\left( \frac{y-b}{\sigma_2} \right) =
%
\frac{1}{\sigma_x}\phi\left( \frac{x-ab}{\sigma_x} \right)
  \frac{1}{\sigma_y}\phi\left( \frac{y-\mu_y}{\sigma_y} \right),
  \label{eq:normaaljaotuste_korrutis}
\end{equation}
\selectlanguage{english}
where
\begin{align*}
\sigma_x^{2}
&= 
\sigma_1^2 + \sigma_2^2 a^2 
&
        \sigma_y &= \frac{\sigma_1 \sigma_2}
          {\sqrt{\sigma_1^2 + \sigma_2^2 a^2}} \\
\mu_y &=
  \frac{\sigma_1^2 b + \sigma_2^2 a x}{\sigma_1^2 + \sigma_2^2 a^2}
\end{align*}

The same in multi-dimensional case:
\begin{multline}
  \frac{1}{\sigma_1}\phi\left( \frac{x_{1} - y}{\sigma_1} \right)
  \frac{1}{\sigma_1}\phi\left( \frac{x_{2} - y}{\sigma_1} \right)
  \dots
  \frac{1}{\sigma_1}\phi\left( \frac{x_{n} - y}{\sigma_1} \right)
  \frac{1}{\sigma_2}\phi\left( \frac{y}{\sigma_2} \right) 
  =
  \\
  =
  \prod_{i=1}^{n}
  \frac{1}{\sigma_1}\phi\left( \frac{x_{i} - y}{\sigma_1} \right)
  \frac{1}{\sigma_2}\phi\left( \frac{y}{\sigma_2} \right) 
  =
  \\
  =
  \frac{1}{\sigma_x}\phi\left( \frac{x_{1}}{\sigma_x} \right)
  \frac{1}{\sigma_x}\phi\left( \frac{x_{2}}{\sigma_x} \right)
  \dots
  \frac{1}{\sigma_x}\phi\left( \frac{x_{n}}{\sigma_x} \right)
  \frac{1}{\sigma_y}\phi\left( \frac{y-\mu_y}{\sigma_y} \right)
  =
  \\
  =
  \prod_{i=1}^{n}
  \frac{1}{\sigma_x}\phi\left( \frac{x_{i}}{\sigma_x} \right)
  \frac{1}{\sigma_y}\phi\left( \frac{y-\mu_y}{\sigma_y} \right)
  \label{eq:normaaljaotuste_korrutis}
\end{multline}
where
\begin{align*}
  \sigma_x^{2}
  &= 
  \sigma_{1}^{2}
  \frac{\sigma_1^2 + n \sigma_2^2}
  {\sigma_1^2 + (n - 1) \sigma_2^2}
  &
  \sigma_y 
  &= 
  \frac{\sigma_1 \sigma_2}
  {\sqrt{\sigma_1^2 + n \sigma_2^2}} 
  \\
  \mu_y 
  &=
  \frac{\sigma_2^2 \sum_{i=1}^{n} x_{i}}{\sigma_1^2 + n \sigma_2^2}
\end{align*}

\selectlanguage{estonian}

\subsection{Jaotuste pered}

\subsubsection{Stabiilne pere}
Mittenegatiivsesse stabiilsesse perre kuuluvad jaotused, mille
momendifunktsioon on
\begin{equation}
M_x (s) = e^{-s^\alpha}, \qquad 0 < \alpha \le 1.
\end{equation}
Stabiilsel pere omadused:
\begin{enumerate}
\item kui juhusliku muutuja $X_i$ jaotusfunktsioon on $G_\alpha$ mis kuulub
  stabiilsesse perre, siis juhusliku muutuja
  \begin{equation}
    Y = n^{-\frac{1}{\alpha}} \sum_{i=1}^N X_i
  \end{equation}
  jaotusfunktsioon on kah $G_\alpha$.
\item Momendifunktsiooni tuletis
  \begin{equation}
    M_x'(s) = -\alpha s^{\alpha - 1} M(s)
  \end{equation}
\end{enumerate}




\newpage
\selectlanguage{english}
\section{Estimators}
\label{sec:estimators}

\subsection{M-Estimators}
\label{sec:M-Estimators}

\subsubsection{Variance}
\label{sec:M-Estimators_variance}

Let an estimator solve
\begin{equation}
  H = \sum_{i} h_{i}(\vec{\hat{\theta}}) = 0.
\end{equation}
From Taylor approximation
\begin{equation}
  \sum_{i} h_{i}(\vec{\hat{\theta}}) 
  = 
  \sum_{i} h_{i}(\vec{\theta}_{0}) 
  +
  \pderiv{\vec{\theta}} 
  \sum_{i} h_{i}(\vec{\theta})\Big|_{\vec{\theta}_{0}}
  (\vec{\hat{\theta}} - \vec{\theta}_{0} )
  = 0
\end{equation}
from where
\begin{equation}
  \vec{\hat{\theta}} - \vec{\theta}_{0} 
  =
  -\left(
    \pderiv{\vec{\theta}} 
    \sum_{i} h_{i}(\vec{\theta})\Big|_{\vec{\theta}_{0}}
  \right)^{-1}    
  \sum_{i} h_{i}(\vec{\theta}_{0}).
\end{equation}
The estimate for variance is
\begin{equation}
  \label{eq:M-estimator_variance}
  \var \vec{\hat{\theta}} =
  \mat{\hat{A}}^{-1} \widehat{\var H} \mat{\hat{A}}^{-1}
\end{equation}
where
\begin{equation}
  \label{eq:M-estimators_A}
  \mat{\hat{A}} = 
  \pderiv{\vec{\theta}} 
  \sum_{i} h_{i}(\vec{\theta})\Big|_{\vec{\theta}_{0}}
\end{equation}
and $\widehat{\var H}$ is an estimator for $\var H$.


\subsection{Maximum likelihood}
\label{sec:maxlik}


\subsubsection{Relationship to Kulback-Leibler distance}

Let the random variables $X_1, X_2, \dots, X_n$ be \iid distributed
according to a distribution function $F(\cdot|\vec{\vartheta})$ and
corresponding density function $
f(\cdot|\vec{\vartheta})$.
Let $f(\cdot|\vec{\vartheta})$ be specified fully parametrically with
a finite unknown parameter vector $\vec{\vartheta}$.  The
\emph{log-likelihood} function of the observed values $x_{1}, x_{2}, \dots,
x_{n}$ is:
\begin{equation}
  \loglik(\vec{\vartheta}|x_1, x_2, \dots, x_n)
  =
  \frac{1}{n}
  \sum_{i=1}^n
  \log f(x_i|\vec{\vartheta}).
\end{equation}
The \emph{maximum likelihood} estimator of $\vec{\vartheta}$ is the
value of $\vec{\vartheta}$ which maximises the log-likelihood
function:
\begin{equation}
  \hat{\vec{\vartheta}}
  =
  \arg \max_{\vec{\vartheta}}
  \loglik(\vec{\vartheta}|x_1, x_2, \dots, x_n).
\end{equation}
This can be written as
\begin{equation}
  \hat{\vec{\vartheta}}
  =
  \arg \max_{\vec{\vartheta}}
  \int \log f(x|\vec{\vartheta}) \,\dif F_n(x),
\end{equation}
where $F_n(x)$ is the empirical distribution function:
\begin{equation}
  F_n(x) 
  =
  \frac{1}{n}
  \sum_{i=1}^n
  \indic(X_i \le x).
\end{equation}
Further, we may write the estimator as
\begin{equation}
  \hat{\vec{\vartheta}}
  =
  \arg \min_{\vec{\vartheta}}
  \left[
    \int \log f(x|\vec{\vartheta}_{0}) \,\dif F_n(x)
    -
    \int \log f(x|\vec{\vartheta}) \,\dif F_n(x)
  \right],
\end{equation}
where $f(\cdot|\vec{\vartheta}_{0})$ is the true density function of
$X$ that does not depend on $\vec{\vartheta}$.  Hence
\begin{equation}
  \hat{\vec{\vartheta}}
  =
  \arg \min_{\vec{\vartheta}}
  \int 
  \log \frac{f(x|\vec{\vartheta}_{0})}{f(x|\vec{\vartheta})} 
  \,\dif F_n(x)
  =
  \arg \min_{\vec{\vartheta}}
  KL(f|\vec{\vartheta}_0, f |\vec{\vartheta})
\end{equation}
where $KL(f,g)$ is Kulback-Leibler distance of two probability density
functions 
\begin{equation}
  KL(f,g) 
  = 
  \E_f 
  \left[
    \log \frac{f(X)}{g(X)}
  \right]
\end{equation}
and $\E_{f}$ means expectation over $X$ that is distributed according
to $f$.




\subsubsection{Information matrix}

Information matrix is defined as
\begin{equation}
  I(\vec{\vartheta})
  \equiv
  - \E \left[ 
    \frac{\partial^2 \loglik(\vartheta)}
    {\partial \vec{\vartheta} \partial \vec{\vartheta}'}
  \right]
  =
  \E \left[
    \pderiv[\loglik(\vec{\vartheta})]
    {\vec{\vartheta}}
    \pderiv[\loglik(\vec{\vartheta})]
    {\vec{\vartheta}'}
  \right]
\end{equation}

\subsection{Entropy Distance}
\label{sec:entropy_distance}

\subsubsection{Entropy Distance}
\label{sec:lin1991_entropy_distance}

\citet{lin1991} defines \emph{entropy distance:}

\begin{equation}
  K(p_{1},p_{2})
  =
  \sum_{x} p_{1}(x)
  \log_{2}
  \frac{p_{1}(x)}
  {\frac{1}{2}[p_{1}(x) + p_{2}(x)]}
  =
  1 + \frac{1}{\log 2}
  \sum_{x} p_{1}(x)
  \log
  \frac{p_{1}(x)}
  {p_{1}(x) + p_{2}(x)}
\end{equation}
Properties: $K(p_{1}, p_{2}) = 0$ if and only if $p_{1} \equiv
p_{2}$.  Otherwise, $K > 0$.

\newpage
\section{Stochastic Processes}

\subsection{Autoregressive (AR) Processes}

\paragraph{AR(1) process}
\selectlanguage{estonian}

Juhuslik muutuja $U$ järgib AR(1) protsessi kui $U$ käesoleva perioodi
realisatsioon on seotud eelmise perioodi omaga
\begin{equation}
  u_t = \varrho u_{t-1} + \varepsilon_t
\end{equation}
ning $\varepsilon_t$ väärtused eri ajaperioodidel on sõltumatud.  Et
protsess oleks stabiilne peab $\varrho$ väärtus jääma vahemikku
$(-1, 1)$. 



\paragraph{AR(2) protsess}

Juhuslik muutuja $U$ järgib AR(2) protsessi kui $U$ käesoleva perioodi
realisatsioon on seotud kahe eelmise perioodi omaga
\begin{equation}
  u_t = \varrho_1 u_{t-1} + \varrho_2 u_{t-2} + \varepsilon_t
\end{equation}
ning $\varepsilon_t$ väärtused eri ajaperioodidel on sõltumatud.


\subsection{Hulkumine}

Definitsioon: hulkumine (\textit{random walk}) on statistiline
protsess
\begin{equation}
z_{t+1} = z_t + \varepsilon_{t+1}.
\end{equation}


\subsubsection{Hulkumine vastu barjääri}
Olgu $z_0=0$ ja $\varepsilon \sim N(0,1)$ \iid protsess.
Siis $z_2$ jaotus tingimusel et $z_1$ es ületa barjääri $\alpha$ on
\begin{equation}
f(z_2|z_1<\alpha) = 
  \frac{1}{\sqrt{2}}
  \frac{\Phi\left(\sqrt{2}\alpha - 
                  \displaystyle\frac{1}{\sqrt{2}}z_2\right)}
    {\Phi(\alpha)}
  \phi \left( \frac{z_2}{\sqrt{2}} \right)
\end{equation}
Tõestus: kirjuta $\phi(x)$ lahti ja integreeri.


\newpage
\section{Statistilised mudelid}
\selectlanguage{english}
\subsection{Tobit-2 model}

Definition:
\begin{align}
  y_{1i}^* &= \vec{z}_i' \vec{\gamma} + u_{1i}
  \\
  y_{2i}^* &= \vec{x}_i' \vec{\beta} + u_{2i}
  \\
  y_{1i} &=
  \begin{cases}
    1, \quad \text{if} \quad y_{1i}^* > 0\\
    0, \quad \phantom{\text{if}} y_{1i}^* \le 0.
  \end{cases}
  \\
  y_{2i} &=
  \begin{cases}
    y_{2i}^*, \quad \text{if} \quad y_{1i}^* > 0\\
    0, \quad \phantom{\text{if}} y_{1i}^* \le 0.
  \end{cases}
\end{align}
Assume
\begin{equation}
  \begin{pmatrix}
    U_{1}\\
    U_{2}
  \end{pmatrix}
 \sim 
 N \left( 
   \begin{pmatrix}
     0 \\ 0
   \end{pmatrix},
  \begin{pmatrix}
  1             & \varrho \\
  \varrho       & \sigma^2
  \end{pmatrix}
\right).
\end{equation}

The Heckman two-step estimator in this case is as follows:
$\vec{\gamma}$ can be consistently estimated with probit model.
Further we may write:
\begin{align}
  \E [Y_2|Y_1 > 0, \vec{x}, \vec{z}] &=
  \vec{x}' \beta + \E [U_2|U_1 > -\vec{z}' \gamma]
  = \vec{x}' \beta + \varrho\sigma\lambda(-\vec{z}' \gamma)
  \\
  \var [Y_2|Y_1 > 0, \vec{x}, \vec{z}] &=
  \E [U_2|U_1 > -\vec{z}' \gamma]
  = \sigma^2 + \varrho^2 \sigma^2
  [-\vec{z}' \gamma \lambda(\vec{z}' \gamma)
  - \lambda^2(\vec{z}' \gamma)]
\end{align}
where $\lambda(x) = \dnorm(x) / \pnorm(x)$; $\pnorm(\cdot)$ and
$\dnorm(\cdot)$ are the normal cumulative distribution function and
density function respectively.  $\varrho$ and $\sigma$ can be
estimated regressing $y_{2i}$ on $\vec{x}_i$ and $\lambda(-\vec{z}'
\gamma)$.  From the coefficient of the latter, $\beta_\lambda$ and the
residual variance $s^2$, one can isolate $\varrho$ and $\sigma$:
\begin{align}
  \hat\sigma^2 &=
  s^2 + \beta_\lambda^2 [\lambda^2(\vec{z}' \gamma) - 
  \vec{z}' \gamma \lambda(\vec{z}' \gamma)]
  \\
  \hat\varrho &=
  \frac{\beta_\lambda}{\hat\sigma}.
\end{align}
Note that $\hat\varrho$ need not to be in $[-1,1]$.

Denote:
\begin{align}
  r &=
  \sqrt{1 - \varrho^2}
  \\
  u_{2i} &= y_{2i} - \vec{x}_i'\vec{\beta}
  \\
  B_i &= 
  \frac{\vec{z}_i' \vec{\gamma} +
    \displaystyle \frac{\varrho}{\sigma} u_{2i}} {r}
  \\
  C(B) &= -\frac{\pnorm(B) \dnorm(B) B + \dnorm(B)^2}{\pnorm(B)^2}
\end{align}
The contribution of observation $i$ to the log-likelihood:
\begin{align}
  \loglik & = 
  \sum_{i:y_{1i} \le 0} 
  \log \pnorm(-\vec{z}_i' \vec{\gamma}) +
  \\
  & +
  \sum_{i:y_{1i} > 0} \left[
    \log \pnorm (B_i) 
    -\frac{1}{2} \log 2\pi - \log \sigma -
    \frac{1}{2} \frac{u_{2i}^2}{\sigma^2} \right].
\end{align}
The gradient of the log-likelihood is:
\begin{align}
                                % dl/dgamma
  \frac{\partial \loglik}{\partial \vec{\gamma}} 
  & = 
  \sum_{i:y_{1i} \le 0} 
  -\lambda(\vec{z}_i' \vec{\gamma}) \vec{z}_i
  +
  \sum_{i:y_{1i} > 0} 
  \lambda(B_i) \frac{\vec{z}_i}{r} \\
                                % dl/dbeta
  \frac{\partial \loglik}{\partial \vec{\beta}} 
  & = 
  \sum_{i:y_{1i} > 0} \left[
    \frac{u_{2i}}{ \sigma^2} -
    \lambda(B_i)
    \frac{\varrho}{\sigma} \frac{1}{r} 
  \right]
  \vec{x}_i
  \\
% dl/dsigma
  \frac{\partial l}{\partial \sigma} 
  & = 
  \sum_{i:y_{1i} > 0} 
  \left[
    \frac{ u_{2i}^2}{\sigma^3} -
    \frac{1}{\sigma} -
    \lambda(B_i) \frac{\varrho}{\sigma^2} 
    \frac{u_{2i}}{r} 
  \right]
  \\
                                % dl/dr
  \frac{\partial \loglik}{\partial \varrho} 
  & = 
  \sum_{i:y_{1i} > 0} 
  \lambda(B_i)
  \frac{ \displaystyle \frac{1}{\sigma}
    u_{2i} + \varrho \vec{z}_i' \vec{\gamma}}
  {r^3}.
\end{align}
Hessian components are
\begin{align}
                                % d2l/ dg dg
  \frac{ \partial^2 \loglik}{\partial\vec{\gamma}\vec{\gamma}'}
  &=
  -\sum_{i:y_{1i} = 0} 
  C(-\vec{z}_i'\vec{\gamma})
  \vec{z}_i \vec{z}_i'
  +
  \sum_{i:y_{1i} = 1} \frac{C(B)}{r}
  \vec{z}_i \vec{z}_i'
  \\
% d2l/ dg db
  \frac{ \partial^2 \loglik}
  {\partial\vec{\gamma} \partial\vec{\beta}'} 
  &=
  - \sum_{i:y_{1i} = 1} 
  C(B) \frac{1}{\sigma} \frac{\varrho}{r}
  \vec{z}_i\vec{x}_i' 
  \\
% d2l/ dg ds3
  \frac{ \partial^2 \loglik}{\partial\vec{\gamma} \partial \sigma} 
  &=
  - \sum_{i:y_{1i} = 1}
  C(B)
  \frac{\varrho u_2}{\sigma^2 r^2}
  \vec{z}_i
  \\
                                % d2l/ dg dr
  \frac{ \partial^2 \loglik}{\partial\vec{\gamma} \partial\varrho} 
  &=
  \sum_{i:y_{1i} = 1} \left[
    C(B) \frac
      {\displaystyle \frac{u2}{\sigma} +
        \varrho \vec{z}_i'\vec{\gamma}}
      {r^4}
    + \lambda(B) \frac{\varrho}{r^3}
    \right] 
    \vec{z}_i 
    \\
                                % d2l / db db
    \frac{ \partial^2 \loglik}
    {\partial\vec{\beta} \partial\vec{\beta}'} 
    &=
    \sum_{i:y_{1i} = 1} 
    \frac{1}{\sigma^2} \left[
      \frac{\varrho^2}{r^2} C(B) - 1 \right]
    \vec{x}_i\vec{x}_i' 
    \\
                                % d2l / db ds
    \frac{ \partial^2 \loglik}
    {\partial\vec{\beta} \partial\sigma} 
    &=
    \sum_{i:y_{1i} = 1} \left[
      C(B) \frac{\varrho^2}{\sigma^3}
      \frac{u_2}{r^2}
      + \frac{\varrho}{\sigma^2}
      \frac{\lambda(B)}{r}
      - 2\frac{u_2}{\sigma^3}
    \right] 
    \vec{x}_i \\
                                % d2l / db dr
    \frac{ \partial^2 \loglik}
    {\partial\vec{\beta} \partial\varrho} 
    &=
    \sum_{i:y_{1i} = 1} \left[
      -C(B) \frac
      {\displaystyle \frac{u_2}{\sigma} +
        \varrho\vec{z}_i'\vec{\gamma}}{r^4}
      \frac{\varrho}{\sigma}
      - \frac{\lambda(B)}{\sigma}
      \frac{1}{r^3}
    \right] 
    \vec{x}_i
    \\
                                % d2l/ ds ds
    \frac{ \partial^2 \loglik}
    {\partial \sigma_3^2} 
    &=
    \sum_{i:y_{1i} = 1} \left[
      \frac{1}{\sigma^2}
      - 3 \frac{ u_2^2}{\sigma^4}
      + 2 \lambda( B)
      \frac{u2}{r}
      \frac{\varrho}{\sigma^3}
      + \frac{\varrho^2}{\sigma^4}
      \frac{ u_2^2}{r^2}
      C(B) 
    \right]
    \\
                                % d2l/ ds_3 dr_3
    \frac{ \partial^2 \loglik}
    {\partial \sigma \partial\varrho} 
    &=
    -\frac{1}{r^3}
    \sum_{i:y_{1i} = 1}  
    \frac{u_2}{\sigma^2} \left[
      C(B) \frac
      {\varrho \left(
          \displaystyle \frac{u_2}{\sigma} +
          \varrho \vec{z}_i'\vec{\gamma} \right)}
      {r}
      + \lambda( B)
    \right]
    \\
                                % d2l/dr dr
    \frac{ \partial^2 \loglik}{\partial \varrho^2} 
    &=
    \sum_{i:y_{1i} = 1}
    \left[
      C(B) \left(
        \frac
        {\displaystyle \frac{u_2}{\sigma}
          + \varrho \vec{z}_i'\vec{\gamma}}
        {r^3}
      \right)^2 + 
      \lambda( B)
      \frac
      {\vec{z}_i'\vec{\gamma}( 1 + 2 \varrho^2) +
        3 \varrho \displaystyle \frac{u_2}{\sigma}}
      {r^5}
    \right]
\end{align}


\subsection{Tobit-2 Model with Binary Outcome}
\label{sec:tobit2B}

The underlying latent model:
\begin{align}
  y_{i}^{S*} &= {\vec{\beta}^{S}}'\vec{x}_i^{S} + \epsilon_{i}^{S}
  \\
  y_{i}^{O*} &= {\vec{\beta}^{O}}'\vec{x}_i^{O} + \epsilon_{i}^{0}
  \\
  y_{i}^{S} &=
  \begin{cases}
    1, \quad \text{if} \quad y_{i}^{S*} > 0\\
    0, \quad \phantom{\text{if}} y_{i}^{S*} \le 0.
  \end{cases}
  \\
  y_{i}^{O} &=
  \begin{cases}
    \text{undetermined}, &\text{if} \quad y_{i}^{S} = 0
       \qquad\text{(case 1)}\\
    0, &\text{if} \quad y_{i}^{O*} \le 0 
       \quad \text{and} \quad y_{i}^{S} = 1
       \quad\text{(case 2)}\\
    1, &\text{if} \quad y_{i}^{O*} > 0 
       \quad \text{and} \quad y_{i}^{S} = 1
       \quad\text{(case 3)}
  \end{cases}
  \label{eq:tobit2Bobservables}
\end{align}
Assume
\begin{equation}
  \begin{pmatrix}
    \epsilon_{1}\\
    \epsilon_{2}
  \end{pmatrix}
 \sim 
 N \left( 
   \begin{pmatrix}
     0 \\ 0
   \end{pmatrix},
  \begin{pmatrix}
  1             & \varrho \\
  \varrho       & 1
  \end{pmatrix}
\right).
\end{equation}

The log-likelihood function contains 3 components, corresponding to
the cases in~\eqref{eq:tobit2Bobservables}:
\begin{align}
  \label{eq:tobit2Bloglik}
  \loglik =
  &\sum_{i \in \text{case 1}}
  \log \Phi \left( 
    -{\vec{\beta}^{S}}'\vec{x}_i^{S}
  \right)\\
  + 
  &\sum_{i \in \text{case 2}} \log\left[
    1 - \Phi \left( 
      -{\vec{\beta}^{S}}'\vec{x}_i^{S}
    \right)
    - \bar \Phi_{2} \left(
      \begin{pmatrix}
        -{\vec{\beta}^{S}}'\vec{x}_i^{S}\\
        -{\vec{\beta}^{O}}'\vec{x}_i^{O}
      \end{pmatrix}
      ,
      \begin{pmatrix}
        1             & \varrho \\
        \varrho       & 1
      \end{pmatrix}
    \right)
  \right]\\
  + 
  &\sum_{i \in \text{case 3}}
  \log\bar \Phi_{2} \left(
      \begin{pmatrix}
        -{\vec{\beta}^{S}}'\vec{x}_i^{S}\\
        -{\vec{\beta}^{O}}'\vec{x}_i^{O}
      \end{pmatrix}
      ,
      \begin{pmatrix}
        1             & \varrho \\
        \varrho       & 1
      \end{pmatrix}
    \right),
\end{align}
where $\bar \Phi_{2}(\cdot,\cdot)$ is the upper tail probability of
2-dimensional normal distribution.

Denote by $\likelihood_{i}$ the corresponding individual likelihood
value.  The score vector:
\begin{align*}
  \pderiv{\vec{\beta}^{S}}\loglik =
  &\sum_{i \in \text{case 1}}
  \frac{1}{\likelihood_{i}}
  \phi \left( 
    -{\vec{\beta}^{S}}'\vec{x}_i^{S}
  \right)
  \vec{x}_{i}^{S}
  \\ + 
  &\sum_{i \in \text{case 2}} 
  \frac{1}{\likelihood_{i}}
    \phi \left( 
      {\vec{\beta}^{S}}'\vec{x}_i^{S}
    \right)
    \bar \Phi \left(
      \frac{
        {\vec{\beta}^{O}}'\vec{x}_i^{O}
        - \varrho{\vec{\beta}^{S}}'\vec{x}_i^{S}
      }
      {\sqrt{1 - \varrho^{2}}}
    \right)
    \vec{x}_{i}^{S}
    \\ + 
  &\sum_{i \in \text{case 3}} 
  \frac{1}{\likelihood_{i}}
    \phi \left( 
      {\vec{\beta}^{S}}'\vec{x}_i^{S}
    \right)
    \Phi \left(
      \frac{
        {\vec{\beta}^{O}}'\vec{x}_i^{O}
        - \varrho{\vec{\beta}^{S}}'\vec{x}_i^{S}
      }
      {\sqrt{1 - \varrho^{2}}}
    \right)
    \vec{x}_{i}^{S}
    \\
    % d/d betaO
    \pderiv{\vec{\beta}^{O}}\loglik =
    &\sum_{i \in \text{case 2}} 
    \frac{1}{\likelihood_{i}}
    \phi \left( 
      {\vec{\beta}^{O}}'\vec{x}_i^{O}
    \right)
    \bar \Phi \left(
      \frac{
        {\vec{\beta}^{S}}'\vec{x}_i^{S}
        - \varrho{\vec{\beta}^{O}}'\vec{x}_i^{O}
      }
      {\sqrt{1 - \varrho^{2}}}
    \right)
    \vec{x}_{i}^{O}
    \\ + 
    &\sum_{i \in \text{case 3}} 
    \frac{1}{\likelihood_{i}}
    \phi \left( 
      {\vec{\beta}^{O}}'\vec{x}_i^{O}
    \right)
    \Phi \left(
      \frac{
        {\vec{\beta}^{S}}'\vec{x}_i^{S}
        - \varrho{\vec{\beta}^{O}}'\vec{x}_i^{O}
      }
      {\sqrt{1 - \varrho^{2}}}
    \right)
    \vec{x}_{i}^{O}
    \\
    % d/d rho
    \pderiv{\varrho}\loglik =
    &-\sum_{i \in \text{case 2}} 
    \phi_{2} \left(
      \begin{pmatrix}
        -{\vec{\beta}^{S}}'\vec{x}_i^{S}\\
        -{\vec{\beta}^{O}}'\vec{x}_i^{O}
      \end{pmatrix}
      ,
      \begin{pmatrix}
        1             & \varrho \\
        \varrho       & 1
      \end{pmatrix}
    \right)
    + \sum_{i \in \text{case 3}} 
    \phi_{2} \left(
      \begin{pmatrix}
        -{\vec{\beta}^{S}}'\vec{x}_i^{S}\\
        -{\vec{\beta}^{O}}'\vec{x}_i^{O}
      \end{pmatrix}
      ,
      \begin{pmatrix}
        1             & \varrho \\
        \varrho       & 1
      \end{pmatrix}
    \right),
\end{align*}
where $\phi_{2}(\cdot,\cdot)$ is 2-dimensional normal density.



\clearpage

\subsection{Tobit-5 Model}

\selectlanguage{estonian}
Definitsioon (lühiduse mõttes on indeks $i$ ära jäetud):
\begin{eqnarray}
y_1^* &=& \vec{Z}'\vec{\gamma} + u_1\\
y_2^* &=& \vec{X}'\vec{\beta_2} + u_2\\
y_3^* &=& \vec{X}'\vec{\beta_3} + u_3\\
y_2 &=& \left\{
 \begin{array}{c@{\quad}l@{\quad}l}
 y_2^*  & \mathrm{kui}  & y_1^* \le 0\\
 0      &               & y_1^* > 0
 \end{array} \right.\\
y_3 &=& \left\{
 \begin{array}{c@{\quad}l@{\quad}l}
 y_3^*  & \mathrm{kui}  & y_1^* > 0\\
 0      &               & y_1^* \le 0
 \end{array}\right.
\end{eqnarray}
Eeldatakse et jääkliikmete jaotus on niisugune:
\begin{equation} \left(
 \begin{array}{c}
 u_1\\
 u_2\\
 u_3
 \end{array} \right) \sim N \left(
 \left( \begin{array}{c}
 0 \\
 0 \\
 0
 \end{array} \right),
 \left( \begin{array}{ccc}
 1                      & \varrho_2\sigma_2     & \varrho_3\sigma_3\\
 \varrho_2\sigma_2      & \sigma^2_2            & \sigma_{23}\\
 \varrho_3\sigma_3      & \sigma_{23}           & \sigma^2_3
 \end{array} \right) \right).
\end{equation}


\subsubsection{Heckmani kahesammuline hinnang}

$\hat\gamma$ leitakse probiti abil. Edasi võib kirjutada
\begin{eqnarray}
y_2 &=& \vec{X}'\vec{\beta_2} - \varrho_2\sigma_2 \lambda(
  -\vec{Z}'\vec{\gamma}) + e_2 \nonumber\\
y_3 &=& \vec{X}'\vec{\beta_3} + \varrho_3\sigma_3 \lambda(
  \vec{Z}'\vec{\gamma}) + e_3
\label{eq:tobit5 H2s}
\end{eqnarray}
Kusjuures
\begin{eqnarray}
\sigma_{e2} &=& \sigma_2^2 \left\{
  1 - \varrho_2^2 \left[
    \lambda^2( -\vec{Z}'\vec{\gamma}) -
    \vec{Z}'\vec{\gamma} \lambda( -\vec{Z}'\vec{\gamma}) \right]
  \right\} \nonumber\\
\sigma_{e3} &=& \sigma_3^2 \left\{
  1 - \varrho_3^2 \left[
    \lambda^2( \vec{Z}'\vec{\gamma}) +
    \vec{Z}'\vec{\gamma} \lambda( -\vec{Z}'\vec{\gamma}) \right] \right\}
\end{eqnarray}
Kui lähendada seost~(\ref{eq:tobit5 H2s}) OLS-ga, siis saab $\lambda$
koefitsendi ja dispersiooni hinnangu abil leida $\hat\varrho$ ja
$\hat\sigma$. Märkus: $\hat\varrho$ ei pruugi olla -1 ja 1 vahel.


\subsubsection{Maksimum-laiklikhuud hinnang}

Mudeli log-laiklihuud on:
\begin{eqnarray}
l &=& -\frac{N}{2}\log 2\pi + \nonumber\\
&+& \sum_{i \in \text{case 2}} \left\{
  -\log\sigma_2
  -\frac{1}{2}\left( \frac{u_2}{\sigma_2} \right)^2
  +\log \pnorm \left[ -\frac
    {\vec{Z}'\vec{\gamma} + \displaystyle \frac{\varrho_2}{\sigma_2}
      \left( y_2 - \vec{X}_i'\vec{\beta}_2 \right)}
    {\sqrt{ 1 - \varrho_2^2}} \right] \right\} \nonumber\\
&+& \sum_{i \in \text{case 3}} \left\{
  -\log\sigma_3
  -\frac{1}{2}\left( \frac{y_3 - \vec{X\beta}_3}{\sigma_3} \right)^2
  +\log \pnorm \left[ \frac
    {\vec{Z'}\vec{\gamma} + \displaystyle \frac{\varrho_3}{\sigma_3}
      \left( y_3 - \vec{X}_i'\vec{\beta}_3 \right)}
    {\sqrt{ 1 - \varrho_3^2}} \right] \right\}. \nonumber\\
&&
\end{eqnarray}
Valikud 2 ja 3 erinevad ainult avaldise märgi poolest funktsiooni
$\pnorm$ sees. Tuletised on:
\begin{eqnarray}
%dl/dg
\frac{\partial l}{\partial \gamma} & = &
  -\sum_2 \frac{\dnorm(B_2)}{\pnorm(B_2)}
    \frac{\vec{Z}}{\sqrt{1-\varrho_2^2}}
  +\sum_3 \frac{\dnorm(B_3)}{\pnorm(B_3)}
    \frac{\vec{Z}}{\sqrt{1-\varrho_3^2}} \\
% dl/db2
\frac{\partial l}{\partial \vec{\beta}_2} & = &
  \sum_2 \left[
    \frac{\dnorm(B_2)}{\pnorm(B_2)} \left(
      \frac{\varrho_2}{\sigma_2}
      \frac{\vec{X}}{\sqrt{1-\varrho_2^2}} \right)
    +\frac{u_2}{\sigma_2^2}\vec{X}
    \right] \\
% dl/dsigma2
\frac{\partial l}{\partial \sigma_2} & = &
  \sum_2 \left[
    -\frac{1}{\sigma_2}
    +\frac{ \left( y_2 - \vec{X}'\vec{\beta}_2 \right)^2}
      {\sigma_2^3}
    +\frac{\dnorm(B_2)}{\pnorm(B_2)}
    \frac{\varrho_2}{\sigma_2^2}
    \frac{y_2 - \vec{X}'\vec{\beta}_2}{\sqrt{1-\varrho_2^2}}
    \right] \\
% dl/droo2
\frac{\partial l}{\partial \varrho_2} & = &
  -\sum_2
    \frac{\dnorm(B_2)}{\pnorm(B_2)}
    \frac{ \displaystyle \frac{1}{\sigma_2}
        ( y_2 - \vec{X}'\vec{\beta}_2) + \varrho_2 \vec{Z}'\vec{\gamma}}
      {( 1 - \varrho_2^2)^{\frac{3}{2}}}\\
% dl/db_3
\frac{\partial l}{\partial \vec{\beta}_3} & = &
  \sum_3 \left[
    -\frac{\dnorm(B_3)}{\pnorm(B_3)} \left(
      \frac{\varrho_3}{\sigma_3}
      \frac{\vec{X}}{\sqrt{1-\varrho_3^2}} \right)
    +\frac{u_3}{\sigma_3^2}\vec{X}
    \right] \\
% dl/dsigma_3
\frac{\partial l}{\partial \sigma_3} & = &
  \sum_3 \left[
    -\frac{1}{\sigma_3}
    +\frac{ \left( y_3 - \vec{X}'\vec{\beta}_3 \right)^2}
      {\sigma_3^3}
    -\frac{\dnorm(B_3)}{\pnorm(B_3)}
    \frac{\varrho_3}{\sigma_3^2}
    \frac{y_3 - \vec{X}'\vec{\beta}_3}{\sqrt{1-\varrho_3^2}}
    \right] \\
% dl/droo_3
\frac{\partial l}{\partial \varrho_3} & = &
  \sum_3
    \frac{\dnorm(B_3)}{\pnorm(B_3)}
    \frac{ \displaystyle \frac{1}{\sigma_3}
        ( y_3 - \vec{X}'\vec{\beta}_3) + \varrho_3 \vec{Z}'\vec{\gamma}}
      {( 1 - \varrho_3^2)^{\frac{3}{2}}}
\end{eqnarray}
Teised tuletised:
\begin{eqnarray}
% d2l/ dg dg
\frac{ \partial^2 l}{\partial\vec{\gamma}^2} &=&
  \sum_2 \frac{C(B_2)}{1-\varrho_2^2}\vec{z}_i \vec{z}_i'
  + \sum_3 \frac{C(B_3)}{1-\varrho_3^2}\vec{z}_i \vec{z}_i'\\
% d2l/ dg db2
\frac{ \partial^2 l}{\partial\vec{\gamma} \partial\vec{\beta}_2'} &=&
  - \sum_2 C(B_2) \frac{1}{\sigma_2} \frac{\varrho_2}{1-\varrho_2^2}
  \vec{Z}\vec{X}' \\
% d2l/ dg ds2
\frac{ \partial^2 l}{\partial\vec{\gamma} \partial \sigma_2} &=&
  - \sum_2
    \frac{\varrho_2 u_2}{\sigma_2^2 ( 1- \varrho_2^2)}
    C(B_2) \vec{Z} \\
% d2l/ dg dr2
\frac{ \partial^2 l}{\partial\vec{\gamma} \partial \varrho_2} &=&
  \sum_2 \left[
    C(B_2) \frac
      {\displaystyle \frac{u_2}{\sigma_2}
        \varrho_2 \vec{Z}'\vec{\gamma}}
      {( 1 - \varrho_2^2)^2}
    - \lambda(B_2) \frac{\varrho_2}
      {(1 - \varrho_2^2)^\frac{3}{2}}
    \right] \vec{Z} \\
% d2l/ dg db3
\frac{ \partial^2 l}{\partial\vec{\gamma} \partial\vec{\beta}_3'} &=&
  - \sum_3 C(B_3) \frac{1}{\sigma_3} \frac{\varrho_3}{1-\varrho_3^2}
  \vec{Z}\vec{X}' \\
% d2l/ dg ds3
\frac{ \partial^2 l}{\partial\vec{\gamma} \partial \sigma_3} &=&
  - \sum_3
    \frac{\varrho_3 u_3}{\sigma_3^2 ( 1- \varrho_3^2)}
    C(B_3) \vec{Z} \\
% d2l/ dg dr_3
\frac{ \partial^2 l}{\partial\vec{\gamma} \partial \varrho_3} &=&
  \sum_3 \left[
    C(B_3) \frac
      {\displaystyle \frac{u_3}{\sigma_3}
        \varrho_3 \vec{Z}'\vec{\gamma}}
      {( 1 - \varrho_3^2)^2}
    + \lambda(B_3) \frac{\varrho_3}
      {(1 - \varrho_3^2)^\frac{3}{2}}
    \right] \vec{Z} \\
% d2l / db_2 db_2
\frac{ \partial^2 l}{\partial\vec{\beta}_2 \partial\vec{\beta}_2'} &=&
  \sum_2 \frac{1}{\sigma_2^2} \left[
  \frac{\varrho_2^2}{1 - \varrho_2^2}C(B_2) - 1 \right]
  \vec{X}\vec{X}' \\
% d2l / db2 ds2
\frac{ \partial^2 l}{\partial\vec{\beta}_2 \partial\sigma_2} &=&
  \sum_2 \left[
    C(B_2) \frac{u_2}{\sigma_2^3}
      \frac{\varrho_2^2}{1 - \varrho_2^2}
    - \frac{\lambda(B_2)}{\sigma_2^2}
      \frac{\varrho_2}{\sqrt{1 - \varrho_2^2}}
    - 2\frac{u_2}{\sigma_2^3}
    \right] \vec{X} \\
% d2l / db_2 dr_2
\frac{ \partial^2 l}{\partial\vec{\beta}_2 \partial\varrho_2} &=&
  \sum_2 \left[
    -C(B_2) \frac
      {\displaystyle \frac{u_2}{\sigma_2} +
        \varrho_2\vec{Z}'\vec{\gamma}}
      {(1 - \varrho_2^2)^2}
      \frac{\varrho_2}{\sigma_2}
    + \frac{\lambda(B_2)}{\sigma_2}
      \frac{1}{(1 - \varrho_2^2)^\frac{3}{2}}
    \right] \vec{X}\\
% d2l / db2 db3
\frac{ \partial^2 l}{\partial\vec{\beta}_2 \partial\vec{\beta}_3}
  &=& 0\\
% d2l / db2 ds3
\frac{ \partial^2 l}{\partial\vec{\beta}_2 \partial\sigma_3}
  &=& 0 \\
% d2l / db2 dr3
\frac{ \partial^2 l}{\partial\vec{\beta}_2 \partial\varrho_3}
  &=& 0 \\
% d2l/ ds_2 ds_2
\frac{ \partial^2 l}{\partial \sigma_2^2} &=&
  \sum_2 \left[
    \frac{1}{\sigma_2^2}
    - 3 \frac{ u_2^2}{\sigma_2^4}
    + \frac{u_2}{\sigma_2^4}
      \frac{\varrho_2^2}{1 - \varrho_2^2}
      C(B_2) \right] - \nonumber\\
&-&   2 \sum_2
      \lambda( B_2)
      \frac{u_2}{\sigma_2^3}
      \frac{\varrho_2}{\sqrt{1- \varrho_2^2}} \\
% d2l/ ds2 dr2
\frac{ \partial^2 l}{\partial \sigma_2 \partial\varrho_2} &=&
  \frac{1}{(1 - \varrho_2^2)^\frac{3}{2}}
  \sum_2  \frac{u_2}{\sigma_2^2} \left[
    -C(B_2) \frac
      {\varrho_2 \left(
      \displaystyle \frac{u_2}{\sigma_2} +
        \varrho_2 \vec{Z}'\vec{\gamma} \right)}
      {\sqrt{1-\varrho_2^2}}
    + \lambda( B_2)
    \right]\\
% d2l / ds2 db3
\frac{ \partial^2 l}{\partial\sigma_2 \partial\vec{\beta}_3}
  &=& 0\\
% d2l / ds2 ds3
\frac{ \partial^2 l}{\partial\sigma_2 \partial\sigma_3}
  &=& 0 \\
% d2l / ds2 dr3
\frac{ \partial^2 l}{\partial\sigma_2 \partial\varrho_3}
  &=& 0 \\
% d2l/dr_2 dr_2
\frac{ \partial^2 l}{\partial \varrho_2^2} &=&
  \sum_2
    C(B_2) \left[ \frac
      {\displaystyle \frac{u_2}{\sigma_2}
        + \varrho_2 \vec{Z}'\vec{\gamma}}
      {( 1 - \varrho_2^2)^\frac{3}{2}}
      \right]^2 - \nonumber\\
&&  - \sum_2 \frac{\dnorm(B_2)}{\pnorm(B_2)}
      \frac
        {\vec{Z}'\vec{\gamma}( 1 + 2 \varrho_2^2) +
          3 \varrho_2 \displaystyle \frac{u_2}{\sigma_2}}
        {(1-\varrho_2^2)^\frac{5}{2}}\\
% d2l / dr2 db3
\frac{ \partial^2 l}{\partial\varrho_2 \partial\vec{\beta}_3}
  &=& 0\\
% d2l / dr2 ds3
\frac{ \partial^2 l}{\partial\varrho_2 \partial\sigma_3}
  &=& 0 \\
% d2l / dr2 dr3
\frac{ \partial^2 l}{\partial\varrho_2 \partial\varrho_3}
  &=& 0 \\
% d2l / db_3 db_3
\frac{ \partial^2 l}{\partial\vec{\beta}_3 \partial\vec{\beta}_3'} &=&
  \sum_3 \frac{1}{\sigma_3^2} \left[
  \frac{\varrho_3^2}{1 - \varrho_3^2}C(B_3) - 1 \right]
  \vec{X}\vec{X}' \\
% d2l / db_3 ds_3
\frac{ \partial^2 l}{\partial\vec{\beta}_3 \partial\sigma_3} &=&
  \sum_3 \left[
    C(B_3) \frac{\varrho_3^2}{\sigma_3^3}
      \frac{u_3}{1 - \varrho_3^2}
    + \frac{\varrho_3}{\sigma_3^2}
      \frac{\lambda(B_3)}{\sqrt{1 - \varrho_3^2}}
    - 2\frac{u_3}{\sigma_3^3}
    \right] \vec{X} \\
% d2l / db_3 dr_3
\frac{ \partial^2 l}{\partial\vec{\beta}_3 \partial\varrho_3} &=&
  \sum_3 \left[
    -C(B_3) \frac
      {\displaystyle \frac{u_3}{\sigma_3} +
        \varrho_3\vec{Z}'\vec{\gamma}}
      {(1 - \varrho_3^2)^2}
      \frac{\varrho_3}{\sigma_3}
    - \frac{\lambda(B_3)}{\sigma_3}
      \frac{1}{(1 - \varrho_3^2)^\frac{3}{2}}
    \right] \vec{X}\\
% d2l/ ds_3 ds_3
\frac{ \partial^2 l}{\partial \sigma_3^2} &=&
  \sum_3 \left[
    \frac{1}{\sigma_3^2}
    - 3 \frac{ u_3^2}{\sigma_3^4}
    + 2 \lambda( B_3)
      \frac{y_3 - \vec{X}'\vec{\beta_3}}{\sqrt{1- \varrho_3^2}}
      \frac{\varrho_3}{\sigma_3^3}
    \right] + \nonumber\\
&&  + \sum_3
      \frac{\varrho_3^2}{\sigma_3^4}
      \frac{ u_3^2}{1 - \varrho_3^2}
      C(B_3) \\
% d2l/ ds_3 dr_3
\frac{ \partial^2 l}{\partial \sigma_3 \partial\varrho_3} &=&
  -\frac{1}{(1 - \varrho_3^2)^\frac{3}{2}}
  \sum_3  \frac{u_3}{\sigma_3^2} \left[
    C(B_3) \frac
      {\varrho_3 \left(
      \displaystyle \frac{u_3}{\sigma_3} +
        \varrho_3 \vec{Z}'\vec{\gamma} \right)}
      {\sqrt{1-\varrho_3^2}}
    + \lambda( B_3)
    \right]\\
% d2l/dr_3 dr_3
\frac{ \partial^2 l}{\partial \varrho_3^2} &=&
  \sum_3
    C(B_3) \left[ \frac
      {\displaystyle \frac{1}{\sigma_3}u_3
        + \varrho_3 \vec{Z}'\vec{\gamma}}
      {( 1 - \varrho_3^2)^\frac{3}{2}}
      \right]^2 + \nonumber\\
&&  + \sum_3 \lambda( B_3)
      \frac
        {\vec{Z}'\vec{\gamma}( 1 + 2 \varrho_3^2) +
          3 \varrho_3 \displaystyle \frac{1}{\sigma_3} u_3}
        {(1-\varrho_3^2)^\frac{7}{2}}
%
\end{eqnarray}
Siin on tähistatud
\begin{eqnarray}
B_2 &=& -\frac
    {\vec{Z}'\vec{\gamma} + \displaystyle \frac{\varrho_2}{\sigma_2}
      \left( y_2 - \vec{X}'\vec{\beta_2} \right)}
    {\sqrt{ 1 - \varrho_2^2}} \\
% B_3
B_3 &=& \frac
    {\vec{Z}'\vec{\gamma} + \displaystyle \frac{\varrho_3}{\sigma_3}
      \left( y_3 - \vec{X}'\vec{\beta_3} \right)}
    {\sqrt{ 1 - \varrho_3^2}}\\
% lambda
\lambda( B) &=& \frac{\dnorm(B)}{\pnorm(B)}\\
% u_2
u_2 &=& y_2 - \vec{X}'\vec{\beta_2}\\
u_3 &=& y_3 - \vec{X}'\vec{\beta_3}\\
% C
C(B) &=& -\frac{\pnorm(B) \dnorm(B) B + \dnorm(B)^2}{\pnorm(B)^2}
\end{eqnarray}


\newpage
\subsection{Kestusmudelid}

Tähistused:
\begin{description}
\item[$\tau$] kestus, algseisundis viibitud aeg
\item[$t$] kalendriaeg
\end{description}


\subsubsection{Kaplan-Meieri hinnang}

\paragraph{KM hinnang diskreetses ajas}
Olgu perioodil $j$ $r_j$ inimest ``riski hulgas'', s.t. $r_j$ inimest
võiksid põhimõtteliselt seisundist lahkuda.  Lahkugu tegelikult $n_j$
inimest, $r_j - n_j$ jäävad edasi algseisundisse.  KM hinnang
hasardile on seega
\begin{align}
  \label{eq:KM_hinnang}
  \hat h &= \frac{n_j}{r_j}\\
  \widehat{\var\hat h_j} &= \frac{\hat h_j(1 - \hat h_j)}{r_j}.
\end{align}
Kui periood $j$ on $k$ kuu pikkune, siis (keskmise) ühe kuu
spetsiifilise hasardi saab
\begin{align}
  \hat\vartheta_j &= 1 - (1 - \hat h_j)^{1/k}\\
                                %
  \widehat{\var\hat\vartheta_j} &=
  \frac{\var\hat h_j}
  {\left[ k(1 - \hat h_j)^{1 - 1/k} \right]^2}
\end{align}

\paragraph{KM hinnang pidevas ajas}
Lahkugu aja $t$ jooksul $r$ algseisundis olnud inimesest $n$.
Keskmine hasart ajaühikus on
\begin{align}
  \hat\vartheta &= -\frac{1}{t} \log(1 - n/r)\\
                                %
  \widehat{\var\hat\vartheta} &=
  \frac{1}{t^2} 
  \frac{n/r}{r - n}.
\end{align}




\subsubsection{Multiplikatiivne mittevaadeldav heterogeensus}

Eeldame et hasart avaldub
\begin{equation}
  \vartheta(\tau|x, v) = \lambda(\tau|x) v
\end{equation}
kus $v$ on mingi kindla jaotusega mittevaadeldav juhuslik suurus.
Nüüd $v$ keskväärtus algseisundisse jääjatel sõltub ajast:
\begin{equation}
  \E (v|T \ge \tau) =
  - \frac{\laplace'[z(\tau|x)]}{\laplace[z(\tau|x)]}
\end{equation}
kus on integreeritud hasart $v$-d arvestamata:
\begin{equation}
  z(\tau|x) = \int_0^\tau \lambda(s|x) \,\dif s
\end{equation}
Kui $v$ on algseisundisse sissevoolus ühikdispersiooniga gammajaotus
parameetriga $\alpha$, siis
\begin{equation}
  \E (v|T \ge \tau) =
  \frac{\alpha}{z(\tau|x) + \alpha^{1/2}}.
\end{equation}



\subsubsection[Tükati konstantne hasart]{Tükati konstantne põhihasart ja diskreetne
mittevaadeldav heterogeensus ning pidev aeg}
\label{sec:MPH_pcw}

\paragraph{Sõltumatud vaatlused}
Eeldatakse, et hasart on konstantne iga $M$ ajavahemiku sees,
erinevatel ajavahemikel võib ta aga olla erinev.  Olgu hasart
kirjeldatud vektoriga $\vec{\lambda}$, kusjuures ajavahemiku $j$
põhihasart olgu $e^{\lambda_j}$.  Mittevaadeldav heterogeensus on
diskreetse jaotusega:
\begin{equation}
v = \begin{cases}
  \begin{array}{ll}
  v_h \equiv 1,\quad &\mbox{tõenäosusega}\quad p_h,\\
  v_l,               &\mbox{tõenäosusega}\quad p_l = 1 - p_h.
  \end{array}
\end{cases}
\end{equation}
Sobiv on parametriseerida
\begin{equation}
  \begin{array}{lcl}
    v_1 & = & \me^{\tilde v_1}\\
    \ldots\\
    v_K & = & 1
  \end{array} 
  \qquad \mbox{ja} \qquad
  \begin{array}{lcl}
    p_1 & = & \Lambda(\tilde p_1)\\
    \ldots\\
    p_K & = & 1 - \sum^{K-1} p_k,
  \end{array}
\end{equation}
Kus $\Lambda(\cdot)$ on logistile jaotusfunktsioon ja $v_k \in \Re$
ning $p_k \in \Re$.

Olgu $m_i$ ajaperiood, mille jooksul inimene lahkub uuritavast
seisundist ja tsenseerimist kirjeldagu $\delta_i = 0$ kui vaatlus on
tsenseeritud ja $1$ kui tsenseerimata ning $\mu_i = \me^{\vec{\gamma}
\vec{x}_i'}$ olgu hasardi inimesest sõltuv osa.  $T_{ij}$ olgu
teadaolev (võimalik et tsenseeritud) aeg, mis inimene $i$ veetis
uuritavas seisundis ajaperioodi $j$ jooksul.  Vektor $\vec{T}_i$ on
$M$-vektor, mille komponendid on $T_{ij}$ ning vector $\vec{d}_i$ on
vektor, mille $j$-s komponent on 1, kui inimene lahkus uuritavast
seisundist ajaperioodil $j$.  Muud komponendid on nullid.

Sel juhul inimese $i$ laiklihuud avaldub:
\begin{multline}
\likelihood_i = p_l \mathcal \likelihood_{li} + p_h \mathcal \likelihood_{hi} = 
%
p_l \left( v_l \mu_i \me^{\lambda_{m_i}} \right)^{\delta_i}
\me^{-z_{li}} +
p_h \left( \mu_i \me^{\lambda_{m_i}} \right)^{\delta_i}
\me^{-z_{hi}},
\end{multline}
kus
\begin{equation}
z_{li} = v_l \mu_i \sum_{j=1}^{M-1} \me^{\lambda_j} T_{ij}
\qquad\mbox{ja}\qquad
z_{hi} = \mu_i \sum_{j=1}^{M-1} \me^{\lambda_j} T_{ij}
\end{equation}
on integreeritud hasart.
Log-laiklihuudi gradient avaldub:
\begin{eqnarray}
% d l / d v_l
\frac{\partial \loglik_i}{\partial v_l} & = &
\frac{p_l \likelihood_{li}}{\likelihood} \left[
  \frac{\delta_i}{v_l} - z_{hi}
  \right] \\
% d l / d p_l
\frac{\partial \loglik_i}{\partial p_l} & = &
\frac{1}{\likelihood_i} \left[ \likelihood_{li} - \likelihood_{hi} \right]\\
% d l / d lambda
\frac{\partial \loglik_i}{\partial \vec{\lambda}} & = &
p_l \frac{\likelihood_{li}}{\likelihood_i} \left(
  \delta_i \vec{d}_i - v_l \mu_i \vec{T}_i \right) +
p_h \frac{\likelihood_{hi}}{\likelihood_i} \left(
  \delta_i \vec{d}_i - \mu_i \vec{T}_i \right)\\
% d l / d gamma
\frac{\partial \loglik_i}{\partial \vec{\gamma}} & = &
\frac{\vec{x_i}}{\likelihood_i} \left[
  p_l \likelihood_l \left( \delta_i - z_{li} \right) +
  p_h \likelihood_h \left( \delta_i - z_{hi} \right) \right]\\
\end{eqnarray}
ja hessi maatriks:
\begin{eqnarray}
% d2 l / d v_l^2
\frac{\partial^2 \loglik_i}{\partial v_l^2} &=&
\frac{\likelihood_{li}}{\likelihood_i} \left[
  \left( \frac{\delta_i}{v_l} - z_{hi} \right)^2
  \left( 1 - p_l \frac{\likelihood_{li}}{\likelihood_i} \right)
  - \frac{\delta_i}{v_l^2} \right]\\
% d2 l / d p_l^2
\frac{\partial^2 \loglik_i}{\partial p_l^2} &=&
- \left( \frac{\likelihood_{li}}{\likelihood_i} - \frac{\likelihood_{hi}}{\likelihood_i} \right)^2\\
% d2 l / d v_l d p_l
\frac{\partial^2 \loglik_i}{\partial v_l \partial p_l} &=&
\frac{\likelihood_{li}}{\likelihood_i}
  \left( \frac{\delta_i}{v_l} - z \right)
  \left( 1 - p_l \frac{\likelihood_{li} - \likelihood_{hi}}{\likelihood_i} \right)\\
% d2 l / d v_l^2
\frac{\partial^2 \loglik_i}{\partial\vec{\lambda} \partial\vec{\lambda}'} &=&
\frac{p_l}{\likelihood_i} \left[
  \frac{\partial \likelihood_{li}}{\partial \lambda}
  \left( \delta_i \vec{d}_i - v_l \mu_i \vec{T}_i \right) -
  \likelihood_{li} \diag
    \left( v_l \mu_i \me^{\vec{\lambda}} \compprod \vec{T}_i \right)
  \right] +\notag\\
&+&
\frac{p_h}{\likelihood_i} \left[
  \frac{\partial \likelihood_{hi}}{\partial \lambda}
  \left( \delta_i \vec{d}_i - \mu_i \vec{T}_i \right) -
  \likelihood_{hi} \diag
    \left( \mu_i \me^{\vec{\lambda}} \compprod \vec{T}_i \right)
  \right] -\notag\\
&-&
\frac{1}{\likelihood_i^2} 
  \left( \frac{\partial \likelihood_i}{\partial \vec\lambda} \right)^2\\
% d2 l / d gamma^2
\frac{\partial^2 \loglik_i}{\partial \vec{\gamma} \partial \vec{\gamma}'} &=&
p_l \frac{\likelihood_{li}}{\likelihood_i}
  \left[ \left( \delta_i - z_{li} \right)^2 - z_{li} \right]
  \vec{x}_i \vec{x}_i' +
p_h \frac{\likelihood_{hi}}{\likelihood_i}
  \left[ \left( \delta_i - z_{hi} \right)^2 - z_{hi} \right] 
  \vec{x}_i \vec{x}_i' - \notag\\
&-&
\frac{\partial \likelihood_i}{\partial \vec{\gamma}}
  \frac{\partial \likelihood_i}{\partial \vec{\gamma}'}\\
% d2 l / d lambda d gamma
\frac{\partial^2 \loglik_i}{\partial \vec{\lambda} \vec{\gamma}'} &=&
\left[ \left( \delta_i - z_{li} \right)
  \left( \delta_i \vec{d}_i - z_{li} \right) - z_{li} \right]
  \frac{\likelihood_{li}}{\likelihood_i} \vec{x}_i' p_l +
\notag\\
&+&
\left[ \left( \delta_i - z_{hi} \right)
  \left( \delta_i \vec{d}_i - z_{hi} \right) - z_{hi} \right]
  \frac{\likelihood_{hi}}{\likelihood_i} \vec{x}_i' p_h -
\frac{1}{\likelihood_i}
  \frac{\partial \likelihood_i}{\partial \vec{\lambda}}
  \frac{\partial \likelihood_i}{\partial \vec{\gamma}'}\\
% d2 l / d lambda d v_l
\frac{\partial^2 \loglik_i}{\partial \vec{\lambda} \partial v_l} &=&
p_l \frac{\likelihood_{li}}{\likelihood_i}
  \left( \frac{\delta_i}{v_l} - z_{hi} \right)
  \left(
    \delta_i \vec{d}_i - 
      v_l \mu_i \me^{\vec{\lambda}} \compprod \vec{T}_i
    - \frac{1}{\likelihood_i} \frac{\partial \likelihood_i}{\partial \vec{\lambda}}
    \right) -
\notag\\
&-&
p_l \frac{\likelihood_{li}}{\likelihood_i} \mu_i 
  \me^{\vec{\lambda}} \compprod \vec{T}_i\\
% d2 l / d lambda d p_l
\frac{\partial^2 \loglik_i}{\partial \vec{\lambda} \partial p_l} &=&
\frac{\likelihood_{li}}{\likelihood_i} \left(
  \delta_i d_i - v_l \mu_i \me^{\vec{\lambda}} \compprod \vec{T}_i
  \right) -
\frac{\likelihood_{hi}}{\likelihood_i} \left(
  \delta_i d_i - \mu_i \me^{\vec{\lambda}} \compprod \vec{T}_i
  \right) -
\notag\\
&-&
\frac{\likelihood_{li} - \likelihood_{hi}}{\likelihood_i}
  \frac{1}{\likelihood_i} \frac{\partial \likelihood_i}{\partial \vec{\lambda}}\\
% d2 l / d gamma d v_l
\frac{\partial^2 \loglik_i}{\partial \vec{\gamma} \partial v_l} &=&
p_l \frac{\likelihood_{li}}{\likelihood_i}
  \left( \frac{\delta_i}{v_l} - z_{hi} \right)
  \left( \delta_i - z_{li} \right) \vec{x}_i -
\notag\\
&-&
p_l \frac{\likelihood_{li}}{\likelihood_i}
  \left( \frac{\delta_i}{v_l} - z_{hi} \right)
  \frac{1}{\likelihood_i} \frac{\partial \likelihood_i}{\partial \vec{\gamma}}
  \vec{x}_i -
p_l \frac{\likelihood_{li}}{\likelihood_i} z_i \vec{x}_i\\
% d2 l / d gamma d p_l
\frac{\partial^2 \loglik_i}{\partial \vec{\gamma} \partial p_l} &=&
\left[
  \frac{\likelihood_{li}}{\likelihood_i}
  \left( \delta_i - z_{li} \right) -
  \frac{\likelihood_{hi}}{\likelihood_i}
  \left( \delta_i - z_{hi} \right)
  \right] \vec{x}_i -
\frac{\likelihood_{li} - \likelihood_{hi}}{\likelihood_i}
  \frac{1}{\likelihood_i} \frac{\partial \likelihood_i}{\partial \vec{\gamma}}
\end{eqnarray}
Eelnevas tähendab $\{\me^{\vec{\lambda}}\}_i = \me^{\lambda_i}$,
$\compprod$ vektorite elementide kaupa korrutamist ($\{\vec{a}
\compprod \vec{b} \}_i = a_i b_i$) ning $\diag\,\vec{a}$ on maatriks,
mille peadiagonaalil on vektor $\vec{a}$ ja mujal nullid.



\newpage
\paragraph{Indiviidi-spetsiif{}iline heterogeensus}
Eeldame nii nagu eespool et hasart on konstantne iga $M$ ajavahemiku
sees, erinevatel ajavahemikel võib ta aga olla erinev.  Avaldugu
hasart
\begin{equation}
  \theta(t|\vec{x},v) = v \me^{\lambda(t)} 
  \me^{\vec{\gamma}
\vec{x}_i'}.
\end{equation}
Mittevaadeldav heterogeensus olgu
diskreetse jaotusega $v \in \{v_1, v_2, \ldots, v_K\}$ ja tõenäosusega
vastavalt $p_1, p_2, \ldots, p_K$.  Olgu inimese mittevaadeldav tunnus
$v$ ajas muutumatu, vaadeldav tunnus aga võib muutuda.  Inimese $i$
spelli $j$ osa laiklihuudi funktsioonis on siis:
\begin{equation}
  \likelihood_{ij}(\cdot|v) = v \theta(t_{ij}|\vec{x}_{ij})^{\delta_{ij}} 
  S(t_{ij}|\vec{x}_{ij},v).
\end{equation}
Siin $t_{ij}$ on spelli vaadeldud kestus, $\delta_{ij}$ on
mitte-tsenseerituse indikaator ja $\vec{x}_{ij}$ on inimese $i$
vaadeldavad isikutunnused spelli $j$ ajal.  $\likelihood_{ij}(\cdot|v)$ on
analoogne indiviidi-spetsiifilise laiklihuudiga sõltumatute vaatluste
juhul. 

Olgu inimese $i$ kohta $N_i$ vaadeldud spelli.  Inimese $i$ osa
laiklihuudis avaldub siis
\begin{equation}
  \likelihood_i(\cdot|v) = \prod_{j=1}^{N_i} \likelihood_{ij}(\cdot|v).
\end{equation}

Vaadeldav laiklihuud avaldub:
\begin{equation}
  \likelihood_i(\cdot) = \sum_{k=1}^K p_k \likelihood_i(\cdot|v_k).
\end{equation}

Log-laiklihuudi gradient avaldub:
\begin{eqnarray}
                                % d l / d v_l
  \frac{\partial \loglik_i}{\partial v_k} & = &
  \frac{p_k}{\likelihood_i(\cdot)}
  \sum_j \frac{\likelihood_i(\cdot|v_k)}{\likelihood_{ij}(\cdot|v_k)}
  \frac{\partial \likelihood_{ij}(\cdot|v_k)}{\partial v_k}\\
                                % d l / d p_l
  \frac{\partial \loglik_i}{\partial p_l} & = &
  \frac{\likelihood_i(\cdot|v_k)}{\likelihood_i(\cdot)}\\
                                % d l / d lambda
  \frac{\partial \loglik_i}{\partial \vec{\lambda}} & = &
  \frac{1}{\likelihood_i(\cdot)} 
  \sum_k p_k
  \sum_j \frac{\likelihood_i(\cdot|v_k)}{\likelihood_{ij}(\cdot|v_k)}
  \frac{\partial \likelihood_{ij}(\cdot|v_k)}{\partial \vec{\lambda}}\\
                                % d l / d gamma
  \frac{\partial \loglik_i}{\partial \vec{\gamma}} & = &
  \frac{1}{\likelihood_i(\cdot)} 
  \sum_k p_k
  \sum_j \frac{\likelihood_i(\cdot|v_k)}{\likelihood_{ij}(\cdot|v_k)}
  \frac{\partial \likelihood_{ij}(\cdot|v_k)}{\partial \vec{\gamma}}.
\end{eqnarray}
$\likelihood_{ij}(\cdot|v)$ tuletised on analoogilised nagu sõltumatute
vaatluste korral.  Lisaks tuleb $p$ järgi gradiendi võtmisel
arvestada, et $\sum p_k = 1$.




\clearpage



\subsubsection{Intervallandmed}
\label{sec:intervallandmed}

Intervallandmetega on tegemist siis kui on vaadeldav ainult fakt et
s"undmus (seisundite vahetamine, tsenseerimine) toimus mingis
kestuse(aja)vahemikus (n"aiteks kuu v~oi n"adala jooksul).  Intervallmudel
sobib ka siis kui ei soovi hasarti t"apselt spetsifitseerida.

Mudel: olgu kestus jagatud $T+1$ vahemikuks: $[0, t_1), [t_1, t_2),
\dots, [t_{T-1}, t_T), [t_T, \infty)$.  Iga indiviidi $i$ kohta olgu
vaadeldav et millises vahemikus ta algsest seisundist lahkus, v~oi et
millises vahemikust vaatlus on tsenseeritud.  Eeldame MPH mudelit nagu
\ref{sec:MPH_pcw}.~{}osas:
\begin{equation}
  \label{eq:MPH}
  \vartheta(\tau|\vec{x}, v) = \lambda(\tau) \me^{\vec{\beta}'\vec{x}} v.
\end{equation}
T~oen"aosus, et isik j"a"ab kogu intervalli $n$ jooksul algseisundisse
avaldub 
\begin{equation}
  S_n(\vec{x}, v) = \exp(-v z_n(\vec{x})) = 
  \exp(-v \me^{\vec{\beta}'\vec{x}} 
  \int_{\tau_{n-1}}^{\tau_n} \lambda(s) \,\dif s).
\end{equation}
N"u"ud v~oib defineerida $\tilde\lambda_n$:
\begin{equation}
  \me^{\tilde\lambda_n}(t_n - t_{n-1})
  \equiv \int_{\tau_{n-1}}^{\tau_n} \lambda(s) \,\dif s,
\end{equation}
kus $\me^{\tilde\lambda_s}$ on keskmine p~ohihasart vahemikus $s$ ja
$\tilde\lambda$ on lihtsalt mudeli parameeter.  Seega p~ohihasart on
spetsifitseeritud mitteparameetriliselt.  Vaatluse laiklihuud
fikseeritud $v$ korral avaldub
n"u"ud
\begin{equation}
  \likelihood(n|\vec{x}, v) =
  \left(
    1 - \me^{-v z_n(\vec{x})}
  \right)^\delta
  \prod_{m=1}^{n-1} \me^{-v z_m(\vec{x})},
\end{equation}
kus $\delta=1$ t"ahendab et vaatlus pole tsenseeritud.  Kogu vaatluse
log-laiklihuud on 
\begin{equation}
  \loglik(n|\vec{x}) = 
  \log\left(
    \sum_{k=1}^K p_k \likelihood(n|\vec{x}, v_k) \right).
\end{equation}

$v$-spetsiifilise laiklihuudi gradient avaldub
\begin{align}
  \pderiv{\vec{\beta}} \likelihood(n|\vec{x}, v) 
  &=
  v E_g(n|\vec{x}, v) \pderiv{\vec{\beta}} z_n(\vec{x}) -
  v S_g(n|\vec{x}, v) \sum_{m=1}^{n-1} \pderiv{\vec{\beta}} z_m(\vec{x})\\
                                %
  \pderiv{\tilde\lambda_s} \likelihood(n|\vec{x}, v) 
  &=
  v E_g(n|\vec{x}, v) \pderiv{\tilde\lambda_s} z_n(\vec{x}) -
  v S_g(n|\vec{x}, v) \sum_{m=1}^{n-1} \pderiv{\tilde\lambda_s} z_m(\vec{x})\\
                                %
  \pderiv{v} \likelihood(n|\vec{x}, v) 
  &=
  E_g(n|\vec{x}, v) z_n(\vec{x}) -
  S_g(n|\vec{x}, v) \sum_{m=1}^{n-1} z_m(\vec{x})
\end{align}
kus
\begin{align}
  E_g(n|\vec{x}, v) 
  &=
  \delta \me^{-v z_n(\vec{x})} \prod_{m=1}^{n-1} \me^{-v z_m(\vec{x})}\\
                                %
  S_g(n|\vec{x}, v)
  &= 
  \likelihood(n|\vec{x}, v) =
  \left(
    1 - \me^{-v z_n(\vec{x})}
  \right)^\delta
  \prod_{m=1}^{n-1} \me^{-v z_m(\vec{x})}
\end{align}
on gradiendi seisundist lahkumise ja seisundis kestmise spetsiifilised
osad ja $z$ gradient avaldub
\begin{align}
  \pderiv{\vec{\beta}} z_n(\vec{x})
  &=
  z_n(\vec{x}) \vec{x}_n\\
                                %
  \pderiv{\tilde\lambda_s} z_n(\vec{x})
  &=
  z_n(\vec{x}) \indic(s = n).
\end{align}
Kogulaiklihuudi gradient on
\begin{align}
  \pderiv{\vec{\beta}} \loglik(n|\vec{x}) 
  &=
  \frac{1}{\likelihood(n|\vec{x})}
  \left(
    \sum_{k=1}^K p_k \pderiv{\vec{\beta}} \likelihood(n|\vec{x}, v)
  \right)\\
                                %
  \pderiv{\tilde\lambda_s} \loglik(n|\vec{x}) 
  &=
  \frac{1}{\likelihood(n|\vec{x})}
  \left(
    \sum_{k=1}^K p_k \pderiv{\tilde\lambda_s} \likelihood(n|\vec{x}, v)
  \right)\\
                                %
  \pderiv{v_k} \loglik(n|\vec{x}) 
  &=
  \frac{1}{\likelihood(n|\vec{x})}
  p_k \pderiv{v_k} \likelihood(n|\vec{x}, v)\\
                                %
  \pderiv{p_k} \loglik(n|\vec{x}) 
  &=
  \frac{\likelihood(n|\vec{x}, v_k)}{\likelihood(n|\vec{x})}
\end{align}




\clearpage
\paragraph{Mitu lõppseisundit}

Mitme lõppseisundiga mudelid esitatakse sageli \emph{kompiiting risk}
kujul.  Too tähendab et kujutatakse ette et kõikidesse
lõppseisunditesse viivad sõltumatud Markovi protsessid, realiseerub
too mille aeg on kõige lühem.  Vajadusel võib tsenseerimist kujutada
ühena lõppseisunditest.

Kui kõik muutujad on vaadeldavad, ei erine mitme lõppseisundiga juhtub
olukorrast kui modelleerida üksikuid lõppseisundeid sõltumatult,
teised seisundid oleksid siis nagu tsenseeritud.  Kui mudelis on
mittevaadeldav heterogeensus, on pilt ainult veidi keerulisem.

Olgu $M$ võimalikku lõppseisundit ja $m \in \{1,\dots,M\}$ olgu
lõppseisundi näitaja.  Olgu seisundisse $m$ ülemineku hasart,
sõltuvalt vaadeldavatest ja mittevaadeldavatest parameetritest,
$\vartheta^m(\tau|\vec{x}, v^m)$.  Mittevaadeldav heterogeensus olgu
$M$-mõõtmelise diskreetse jaotusega: $v^m \in \{v_1^m, \dots,
v_{K^m}^m$ kusjuures $\vec{v} = (v_{k^1}^1, \dots, v_{k^M}^M)$ esineb
tõenäosusega $p_{k^1 \dots k^M}$.  Eeldame et erinevate spellide
jooksul on $\vec{v}$ konstantne, $\vec{x}$ aga võib muutuda.

Inimese $i$ spelli $j$, osa laiklihuudi funktsioonis siirde $m$ järgi
on nüüd:
\begin{equation}
  \likelihood_{ij}^m(\cdot|v^m) = 
  \left[ \vartheta^m(\tau_{ij}|\vec{x}_{ij}) \right]^{\delta_{ij}^m} 
  S^m(\tau_{ij}|\vec{x}_{ij},v^m).
\end{equation}
Siirdega seotud liige $\left[ \vartheta^m(\tau_{ij}|\vec{x}_{ij})
\right]^{\delta_{ij}^m}$ ja püsimisega seotud liige
$S^m(\tau_{ij}|\vec{x}_{ij},v^m)$ avalduvad nii nagu ühe spelli ja ühe
lõppseisundi puhul (vaata näiteks~\ref{sec:MPH_pcw}
või~\ref{sec:intervallandmed}).  Siin $\tau_{ij}$ on spelli vaadeldud
kestus, $\delta_{ij}^m$ on mitte-tsenseerituse indikaator $m$-seisundi
mõttes ($=1$ kui läks seisundisse $m$ ja 0 kui es lähe) ja
$\vec{x}_{ij}$ on inimese $i$ vaadeldavad isikutunnused spelli $j$
ajal.  Spelli kogulaiklihuud on:
\begin{equation}
  \likelihood_{ij}(\cdot|\vec{x}_{ij},\vec{v}) = 
  \prod_{m=1}^M
  \likelihood_{ij}^m(\cdot|\vec{x}_{ij}, v^m).
\end{equation}
Kompiiting risk mudel eeldab et siirded on sõltumatud (kui kontrollida
$\vec{x}$ ja $\vec{v}$ suhtes), seega siis laiklihuudi korrutis
siirete kaupa.  

Olgu inimese $i$ kohta $N_i$ vaadeldud spelli.  Inimese $i$ osa
laiklihuudis avaldub siis
\begin{equation}
  \likelihood_i(\cdot|\vec{x}, \vec{v}) = 
  \prod_{j=1}^{N_i} \likelihood_{ij}(\cdot|\vec{x}, \vec{v}).
\end{equation}
ja vaadeldav laiklihuud avaldub:
\begin{equation}
  \likelihood_i(\cdot|\vec{x}) = 
  \sum_{k^1=1}^{K^1} \dots \sum_{k^M=1}^{K^M} 
  p_{k^1 \dots k^M} 
  \likelihood_i(\cdot|\vec{x}, \vec{v}).
\end{equation}

Laiklihuudi gradient avaldub:
\begin{align}
                                % d l / d lambda
  \frac{\partial \likelihood_i(\cdot|\vec{x})}
  {\partial \vec{\lambda}^m} & = 
  \sum_{k^1=1}^{K^1} \dots \sum_{k^M=1}^{K^M} 
  p_{k^1 \dots k^M} 
  \likelihood_i(\cdot|\vec{x}, \vec{v})
  \sum_{j=1}^{N_i} 
  \left[ 
    \frac{1}{\likelihood_{ij}^m(\cdot|\vec{x} v_{k^m}^m)} 
    \frac{\partial \likelihood_{ij}^m(\cdot|\vec{x} v_{k^m}^m)}
    {\partial \vec{\lambda}^m} 
  \right]
    \\
                                % d l / d gamma
  \frac{\partial \likelihood_i(\cdot|\vec{x})}
  {\partial \vec{\gamma}^m} & = 
  \sum_{k^1=1}^{K^1} \dots \sum_{k^M=1}^{K^M} 
  p_{k^1 \dots k^M} 
  \likelihood_i(\cdot|\vec{x}, \vec{v})
  \sum_{j=1}^{N_i} 
  \left[ 
    \frac{1}{\likelihood_{ij}^m(\cdot|\vec{x} v_{k^m}^m)} 
    \frac{\partial \likelihood_{ij}^m(\cdot|\vec{x} v_{k^m}^m)}
    {\partial \vec{\gamma}^m} 
  \right]
    \\
                                % d l / d v_l
  \frac{\partial \likelihood_i(\cdot|\vec{x})}
  {\partial v_k^m} & = 
  \sum_{k^1=1}^{K^1} \dots \sum_{k^{m-1}=1}^{K^{m-1}} 
  \sum_{k^{m+1}=1}^{K^{m+1}} \dots \sum_{k^M=1}^{K^M} 
  p_{k^1 \dots k^{m-1} k k^{m+1} \dots k^M} 
  \cdot
  \notag\\
  & \cdot
  \likelihood_i(\cdot|\vec{x}, \vec{v})
  \sum_{j=1}^{N_i} 
  \left[ 
    \frac{1}{\likelihood_{ij}^m(\cdot|\vec{x} v_{k}^m)} 
    \frac{\partial \likelihood_{ij}^m(\cdot|\vec{x} v_k^m)}
    {\partial \vec{\gamma}^m} 
  \right]
    \\
                                % d l / d p_l
  \frac{\partial \likelihood_i(\cdot|\vec{x})}
  {\partial p_{k^1 \dots k^M}}
    & = 
  \likelihood_i(\cdot|\vec{x}, (v_{k^1}^1, \dots, v_{k^M}^M))
\end{align}
$\likelihood_{ij}^m(\cdot|\vec{x}, \vec{v})$ tuletised on nii nagu
sõltumatute vaatluste korral.




\clearpage
\subsubsection{Parametriseerimine}

\selectlanguage{english}
\paragraph{Logistic transition probability in discrete-time}
In discrete time, it is useful to parameterise the
destination-specific exit probabilities logistically as
\begin{equation}
  p^m \equiv \Pr(\text{exit to destination $m$}) = 
  \frac{\me^{\lambda^m}}{1 + \sum_k \me^{\lambda^k}}.
\end{equation}
The probabilities are guaranteed to be positive and their sum to be
less than one while $\lambda$-s may be unbounded.  The components of
the corresponding gradient transformation matrix are
\begin{equation}
  \pderiv{\lambda^k} p^m =
  \indic(k = m) p_m - p_k p_m
\end{equation}
and the inverse transformation
\begin{equation}
  l^m = \log \frac{p^m}{1 - \sum_k p^k}.
\end{equation}



\paragraph{Transition probability in continuous time}
A good choice for parametrising the time-dependent part of the hazard
is
\begin{equation}
  \lambda = \me^{\tilde \lambda}
\end{equation}
The $\lambda$ is now guaranteed to be positive.


\selectlanguage{estonian}
\paragraph{Diskreetne mittevaadeldav heterogeensus}
Diskreetne mittevaadeldav heterogeensus, üks lõppseisund: $v \in
\{v_1, v_2,\dots,v_K$ ja vastavad tõenäousused $p_1, p_2, \dots,
p_K$.  

Kui laiklihuudi maksimeerimisel kasutada vastavaid parameetreid
$\tilde v$ ning $\tilde p$ ($\tilde N$ parameetrit) kuid laiklihuudi
arvutamiseks nad normaalkujule ($N$ parameetrit) teisendada, siis peab
arvestama, et vastavad gradiendi komponendid teisenevad:
\begin{equation}
  \begin{pmatrix}
    \displaystyle\pderiv{\tilde{\vec{v}}}\loglik_i\\[2ex]
    \displaystyle\pderiv{\tilde{\vec{p}}}\loglik_i
  \end{pmatrix}
  =
  \begin{pmatrix}
    \displaystyle\pderiv{\tilde{\vec{v}}}v &
    \displaystyle\pderiv{\tilde{\vec{v}}}p\\[2ex]
    \displaystyle\pderiv{\tilde{\vec{p}}}v &
    \displaystyle\pderiv{\tilde{\vec{p}}}p
  \end{pmatrix}
  \begin{pmatrix}
    \displaystyle\pderiv{v}\loglik_i\\[2ex]
    \displaystyle\pderiv{p}\loglik_i
  \end{pmatrix}
  \equiv
  \mat{C}
  \begin{pmatrix}
    \displaystyle\pderiv{v}\loglik_i\\[2ex]
    \displaystyle\pderiv{p}\loglik_i
  \end{pmatrix}
  \label{eq:kestus_parametriseerimine}
\end{equation}
Maatriksi $\mat{C}$ ridade arv vastab algkuju parameetrite arvule
$\tilde N$ ning veergude arv normaalkuju parameetrite arvule $N$ (s.h.
lineaarselt s\~{o}ltuvad $p_H$ ja $v_H$).  Kovarjatsjoonimaatriksi $p$
ning $v$ sisaldav osa on vastavalt:
\begin{equation}
  \mat\Sigma = \mat C' \tilde{\mat\Sigma} \mat C
\end{equation}


\paragraph{Heterogeensus} 
Tõenäosused on võimalik parametriseerida kui
\begin{equation}
  p_k = \frac{\displaystyle \me^{\tilde p_k}}
  {\displaystyle \sum_{k = 1}^{K} \me^{\tilde p_k}}
  \qquad\mbox{ja}\qquad 
  \sum_{k=1}^K \tilde p_k = 0,
\end{equation}
kus $K$ on jaotuse toetuspunktide arv.

\subparagraph{Keskväärtuse normeerimine} $v$ keskväärtuse võib
normeerida üheks:
\begin{equation}
  v_k = \me^{\tilde v_k}
  \qquad\mbox{ja}\qquad 
  \sum_{k=1}^K p_k v_k = 1.
\end{equation}
Vastav p\"{o}\"{o}rdteisendus t\~{o}en\"{a}osuste tarvis on
\begin{equation}
  \tilde p_k = \log p_k - 
  \frac{\sum_{l=1}^K \log p_l}{K}.
\end{equation}

$\mat{C}$ komponendid on:
\begin{align}
 \pderiv{\tilde{v}_l}v_k
  &=
  \begin{cases}
    v_k & \text{kui\qquad} k = l < K\\
    -v_l \displaystyle\frac{p_l}{p_K} & \text{kui\qquad} k = K\\
    0             & \text{muudel juhtudel}
  \end{cases}\\
                                %
  \pderiv{\tilde{v}_l}p_k
  &=
  0\\
                                %
  \pderiv{\tilde{p}_l}v_k
  &=
  \begin{cases}
    \displaystyle\frac{1}{p_K}
    \left[
      (1 - p_K) + p_K v_K + (1 - p_l)v_l
    \right]
    & \text{kui\qquad} k = K\\
    0 & \text{muudel juhtudel}
  \end{cases}\\
                                %
  \frac{\partial p_k}{\partial \tilde p_l} 
  &= 
  -p_k(p_l - p_K) 
  +
  \indic(k=l) p_K
  -
  \indic{(K=k)}
  p_K
\end{align}



\subparagraph{$v$ komponendi normeerimine}
Defineerime
\begin{equation}
  v_K = 1.
\end{equation}
Näib et nii on intervallandmete juures tulemused mõnevõrra
stabiilsemad, samas on põhihasarti raskem tõlgendada.

$\mat{C}$ komponendid on
\begin{align}
 \pderiv{\tilde{v}_l}v_k
  &=
  \begin{cases}
    v_k & \text{kui\qquad} k = l < K\\
    0             & \text{muudel juhtudel}
  \end{cases}\\
                                %
  \pderiv{\tilde{v}_l}p_k
  &=
  0\\
                                %
  \pderiv{\tilde{p}_l}v_k
  &=
  0\\
                                %
  \frac{\partial p_k}{\partial \tilde p_l} 
  &= 
  -\frac{\displaystyle \me^{\tilde p_k} 
    \left(\me^{\tilde p_l} - \me^{\tilde p_H} \right)}
  {\left( \sum_{i=1}^H \me^{\tilde p_i} \right)^2} +
  \indic(k=l) 
  \frac{\me^{\tilde p_k}}{ \sum_{i=1}^H \me^{\tilde p_i}}
  +
  \indic{(K=k)}
  \frac{\me^{\tilde{p}_K}}
  {\sum_{i=1}^K \me^{\tilde{p}_i}}
\end{align}



\paragraph{$M$ lõppseisundit}

\eqref{eq:kestus_parametriseerimine} asemel võib nüüd kirjutada:
\begin{multline}
  \begin{pmatrix}
    \displaystyle\pderiv{\tilde{\vec{v}}^1}\loglik_i\\[2ex]
    \displaystyle\pderiv{\tilde{\vec{v}}^2}\loglik_i\\[2ex]
    \dots\\[2ex]
    \displaystyle\pderiv{\tilde{\vec{v}}^M}\loglik_i\\[2ex]
    \displaystyle\pderiv{\tilde{\vec{p}}}\loglik_i
  \end{pmatrix}
  =
  \begin{pmatrix}
    \displaystyle\pderiv{\tilde{\vec{v}}^1} v^1 &
    \displaystyle\pderiv{\tilde{\vec{v}}^1} v^2 &
    \dots &
    \displaystyle\pderiv{\tilde{\vec{v}}^1} v^M &
    \displaystyle\pderiv{\tilde{\vec{v}}^1}p\\[2ex]
                                %
    \displaystyle\pderiv{\tilde{\vec{v}}^2} v^1 &
    \displaystyle\pderiv{\tilde{\vec{v}}^2} v^2 &
    \dots &
    \displaystyle\pderiv{\tilde{\vec{v}}^2} v^M &
    \displaystyle\pderiv{\tilde{\vec{v}}^2}p\\[2ex]
                                %
    \cdots &
    \cdots &
    \ddots &
    \cdots &
    \cdots\\[2ex]
                                %
    \displaystyle\pderiv{\tilde{\vec{v}}^M} v^1 &
    \displaystyle\pderiv{\tilde{\vec{v}}^M} v^2 &
    \dots &
    \displaystyle\pderiv{\tilde{\vec{v}}^M} v^M &
    \displaystyle\pderiv{\tilde{\vec{v}}^M}p\\[2ex]
                                %
    \displaystyle\pderiv{\tilde{\vec{p}}} v^1 &
    \displaystyle\pderiv{\tilde{\vec{p}}} v^2 &
    \dots &
    \displaystyle\pderiv{\tilde{\vec{p}}} v^M &
    \displaystyle\pderiv{\tilde{\vec{p}}} p
  \end{pmatrix}
  \begin{pmatrix}
    \displaystyle\pderiv{\vec{v}^1}\loglik_i\\[2ex]
    \displaystyle\pderiv{\vec{v}^2}\loglik_i\\[2ex]
    \dots\\[2ex]
    \displaystyle\pderiv{\vec{v}^M}\loglik_i\\[2ex]
    \displaystyle\pderiv{\vec{p}}\loglik_i
  \end{pmatrix}
  \\
  \equiv
  \mat{C}
  \begin{pmatrix}
    \displaystyle\pderiv{\vec{v}^1}\loglik_i\\[2ex]
    \displaystyle\pderiv{\vec{v}^2}\loglik_i\\[2ex]
    \dots\\[2ex]
    \displaystyle\pderiv{\vec{v}^M}\loglik_i\\[2ex]
    \displaystyle\pderiv{\vec{p}}\loglik_i
  \end{pmatrix},
\end{multline}
kus $\mat{C}$ on $\tilde N \times N$ maatriks.

Olgu $p_{Dk}^m$ suuna $m$ spetsiifiline tõenäosus, s.t. millise
tõenäosusega esineb väärtus $v_k^m$.  Tuletised võib nüüd avaldada kui
\begin{equation}
  \pderiv{\tilde{p}} v^m = 
  \pderiv{\tilde{p}} p
  \pderiv{p} p_D^m
  \pderiv{p_D^m} v^m
\end{equation}



\clearpage
\selectlanguage{english}
\section{Algorithms}
\label{sec:algorithms}

\subsection{Arrays}



\paragraph{Array indexing}

Let $\mat{M}$ be a $N$-dimensional column-major zero-based array, and $m[i_1, i_2, \dots,i_N]$
its element where $i_j \in \{0,\dots,K_j-1\}$ and $K_j$ is the size of
dimension $j$.  Vectorized index for element $m[i_1, i_2, \dots,i_N]$ is
\begin{equation}
  j = i_1 + K_1 i_2  + K_2 i_3 + \dots + K_{N-1} i_N.
\end{equation}
The array indices can be recovered from the vector index $j$ as: 
\begin{align*}
  i_1 &= j \mod K_1\\
  i_2 &= [j \mod (K_1 K_{2})] \idiv K_1\\
  i_3 &= [j \mod (K_1 K_2 K_{3})] \idiv (K_1 K_{2})\\
  i_4 &= [j \mod (K_1 K_2 K_3 K_{4})] \idiv (K_{1} K_{2} K_{3})\\
      & \dots
\end{align*}
where $\idiv$ is the integer division.



\newpage
\bibliographystyle{apecon}
\bibliography{Economics}

\end{document}
